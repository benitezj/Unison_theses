\begin{abstract}
    \begin{center}
    {\huge Resumen}
\end{center}
Se describe el diseño y el rendimiento esperado del Tracker End-cap Pixel (TEPX) y otros sub detectores como luminómetros para el experimento CMS, en el próximo LHC de alta luminosidad (HL-LHC). El detector TEPX está compuesto por 4 discos de doble cara que cubren un rango en $ | z | $ de 175 a 265 cm, cada uno compuesto por 5 anillos hechos de sensores de silicio con un alto número de píxeles. El Disco 4 Anillo 1 (TEPXD4R1) se designará solo para mediciones de luminosidad, utilizando el mismo método que el luminómetro TEPX, el método de conteo de conjuntos de píxeles (PCC). Para la etapa del HL-LHC, el objetivo es lograr una incertidumbre final de $ 1 \% $ para las mediciones de luminosidad. Se presentan el rendimiento esperado y la linealidad de TEPX, TEPXD4R1 y otros luminómetros en términos de precisión estadística, para la calibración de escaneo de van der Meer (vdM) y para condiciones físicas. El ancho de banda necesario para la transferencia de los datos de luminosidad y el disco necesario para su almacenamiento es estimado.

\end{abstract}