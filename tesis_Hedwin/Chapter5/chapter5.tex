\chapter{Conclusion and Summary }
With the new luminosity goal of the HL-LHC, the CMS detector will receive several upgrades, in order to keep up with the new luminosity demands. These upgrades will come to several systems, which includes various subdetectors and luminometers. One of these subdetectors TEPX,
will perform luminosity measurements with high statistics. 
One of its rings, TEPXD4R1, will perform luminosity measurements only. The TEPX will be operated by BRIL, utilizing the pixel cluster counting (PCC) method, as the main source of luminosity measurements.

The PCC algorithm will take advantage of the low occupancy and the 800 million pixels distributed across the 2 $m^2$ of the TEPX luminometer, to provide precise luminosity measurements. The TEPX luminometers will provide event rate measurements utilizing two types of objects: pixel clusters and two-fold coincidences. 

In this work, the statistical precision of the TEPX luminometers, and other luminometers, is studied using simulated data. For physics runs (pileup of 200), the TEPX luminometers achieved a statistical precision per bunch per 1 s, below $0.1\%$ for pixel clusters and below $0.4\%$ for two-fold coincidences. In vdM conditions (pileup of 0.5), the precision is below $0.4\%$ per bunch per 30 s, for pixel clusters and below $0.8\%$ for coincidences. This translates to a precision under $0.1\%$ per bunch for the calibration constant ($\sigma_{vis}$) for pixel clusters, and below $0.3\%$ for coincidences. Both luminometers also achieve linearity deviations well below $1\%$ up to a pileup of 200. 
These results show an excellent expected performance for TEPX luminometers, and in combination with the rest of the luminometers (OT layer 6, DT and BMTF), will provide precise luminosity measurements for the HL-LHC phase.

The network bandwith and disk storage needed for the BRIL Phase 2 luminosity systems has been also estimated. Concluding that a bandwith of at least 18.3 Mbps is required, and a disk storage of 280.46 GB per day is needed.