\chapter{Summary}
\label{ch4}

This work aims to focus on the calibration of hte luminosity for the PCC luminometer base on the Van Der Meer scans conducted by the CMS collaboration on 2024 for the fill 9639 in the proton proton collisions of the LHC at $\sqrt{s}$ 13.6 TeV obtaining the calibration value from seven scan pairs (5 vdM's and 2 BI's), the concept of luminosity, the LHC, CMS, PCC luminometer and calibration method to estimate $\sigma_{visible}$ were covered in this work.

This works also explains the reprocess of the ALCARECO dataset to then using a veto list to select the good modules of the pixel detector to remove any inastabilitie in the measurement of the luminosity. Also more data were processed obtaining a value for the background analyzing the five super separation scans to correct the rates of the PCC luminometer. The model used for this calibration was QG function, and 3 corrections were applied to fhe final value of the calibration, the background correction, the ghost satellite correction, the beam beam and dynamic beta corrections.

The obtained value for the calibration constant was $\sigma_{visible}$ = 1460890 \pm 546$ the analysis and results reported in this work will be useful for an analysis of the pre eliminary luminosity measurement for Run 3 in the CMS experiment of the lHC, where the rest of the luminometers and It's certainties will be studied along with this one.