
\thispagestyle{empty}
 \vspace*{2.2cm}
\begin{center}
\Large \textbf{Resumen}
\end{center} 
\vspace{1.5cm}

%\renewcommand{\absnamepos}{empty} % originally center
%\begin{center}
%	\justifying
Se presenta un estudio sobre la producci\'on del boson de Higgs y un top quark ($tH$) en el canal de 2 muones con el mismo signo $\mu^\pm \mu^\pm$ usando datos publicados por el experimento CMS en el CERN.
Estudiando este proceso se explora este mecanismo de producci\'on del boson de Higgs que a\'un no se ha detectado experimentalmente.
La sensibilidad esperada es calculada usando  datos de tipo Asimov para 35.9 fb$^{-1}$ de colisiones de prot\'on-prot\'on y es extrapolada la fase de alta luminosidad del LHC (HL-LHC).
La sensibilidad de la se\~nal es tambi\'en estudiada para el modelo con acoplamiento de Yukawa invertido $k_t$=-1, donde $k_t$ es el modificador del par\'ametro de acoplamiento de top-Higgs, en comparaci\'on con el Modelo Est\'andar. Para el modelo SM, se obtuvo un l\'imite de 17 en la fuerza de la se\~nal a un nivel de confianza de 95$\%$ con respecto al valor esperado usando 35.9 fb$^{-1}$ de luminosidad integrada, mientras que con 3000 fb$^{-1}$ se obtiene un l\'imite de 4.3. Para el modelo modificado con $k_t$=-1, se obtiene un l\'imite de 2.3 usando 35.9 fb$^{-1}$ , mientras que con 3000 fb$^{-1}$  se podr\'ia medir la fuerza de una posible se\~nal con 10$\%$ de incertidumbre.
%\end{center}
