\documentclass[final,3p]{CSP}
\usepackage{amssymb}
\usepackage{changepage}
\usepackage{float}
\usepackage{hyperref}
\usepackage{url}
\usepackage{afterpage}
\usepackage{natbib}
\usepackage{setspace}
\usepackage{fancyhdr}
\pagestyle{fancy}
\fancyhf{}

\def\Student{Ashish Sehrawat}
\def\Title{THESIS PROJECT PROPOSAL}
\def\Prog{Doctorado en Ciencias (F\'{i}sica) }
\def\Dept{Departamento de Investigac\'{i}on en Fis\'{i}ca}
\def\Institution{Universidad de Sonora}
\def\Director{Dr. Jos\'{e} Feliciano Ben\'{i}tez Rubio}
\def\ProjectTitle{Measurement of the luminosity in proton-proton collisions at the CMS experiment of the Large Hadron Collider}
\def\ResearchLine{Astrof\'{i}sica, Cosmolog\'{i}a y F\'{i}sica de Part\'{i}culas}


\newcommand{\SubItem}[1]{
    {\setlength\itemindent{15pt} \item[-] #1}
}


%%header and footer
\lhead{\Student}
\rhead{\Title}
\lfoot{\Dept}
\rfoot{Page \thepage}
\setlength{\headsep}{0.2in}
\renewcommand{\footrulewidth}{0.4pt}% default is 0pt


\begin{document}

%%%%Title Page
\begin{titlepage}
  \centering
  \hspace{0pt}
  \vfill
      {\scshape\Large \Title \par}
      revision requested december 2020
      
      \vspace{2cm}
      %\begin{adjustwidth}{2cm}{2cm}
       {
         TITLE:\par
         {\large \ProjectTitle \par}
       }
       %\end{adjustwidth}
       
       \vspace{0.5cm}
       %\begin{adjustwidth}{2cm}{2cm}
       {
         RESEARCH LINE: \par
         \ResearchLine \par
       }
       %\end{adjustwidth}

        
       \vspace{4cm}
       {\underline{\hspace{8cm}}\par}
       {\scshape \Student \par}
       {Student\par}

       \vspace{1cm}
       {\underline{\hspace{8cm}}\par}
       {\scshape \Director \par}
       {Director\par}

       \vspace{1cm}
       {\Prog \par}
       {\Dept \par}
       {\Institution \par}

       \vspace{4cm}
       {\today}

\hspace{0pt}
\vfill

\end{titlepage}


%%%%% white page for print out
\shipout\null


%%%% Abstract Page
\newpage
\hspace{2pt}
\vfill

\begin{adjustwidth}{1cm}{1cm}

  \begin{center}
    {\Large \ProjectTitle \par}
    \vspace{1cm}
    {\itshape\textbf{Abstract}\par}
  \end{center}
  
  \vspace{1 cm}
 
  
  \onehalfspacing
  This research project proposes ...

\end{adjustwidth}

\hspace{2pt}
\vfill

%%%%%% Begin the body
\newpage
\section{BACKGROUND}


\onehalfspacing
The standard model (SM) of particle physics is so far the best theoretical model to describe the interaction of elementary particles using three of the four fundamental forces of nature which are electromagnetic force, strong nuclear force and the weak nuclear force.
%Gravitational force is neglected as the strength of this force is very weak at the scales over which elementary particle interact with each other.
The SM is divided into two categories, bosonic sector and fermionic sector.
The bosonic sector contain particles which mediate the fundamental forces of  nature and the fermionic sector contain particles which make up all the matter in our universe.
There are three generations of fermion particles: the first generation consists of up (u) quark, down (d) quark, electron and electron neutrino, the second generation consist of charm (c) quark, strange (s) quark, muon and muon neutrino, and the third generation has the top (t) quark, bottom (b) quark, tau and tau neutrino.
The bosonic sector consists of the gauge bosons: gluon, photon, $W^{\pm}$, $Z^0$ which mediate strong nuclear force, electromagnetic force and weak nuclear force respectively.
The Higgs boson ($H$), is the last of the gauge bosons, it gives mass to the other particles via electroweak symmetry breaking mechanism \cite{Chatrchyan:2012xdj}.
%All the standard model particles are shown in Figure 1.
%Higgs boson can be produced at the particle colliders like the Large Hadron Collider (LHC) in Geneva, Switzerland.



%The Standard Model (SM) of particle physics has successfully described most of the experimental data till now but a very large number of free-parameters and the fine tuning related to these parameters suggest new physics beyond SM as mentioned below.
%\begin{itemize}
%\item{Fine structure constant $\alpha$},
%\item{Weinberg angle or the weak mixing angle $\theta_W$},
%\item{The coupling constant of strong interaction $\alpha_s$},
%\item{Electroweak symmetry breaking energy scale v},
%\item{ Higgs potential self coupling $\lambda$ or the Higgs mass $m_H$},
%\item{Three weak mixing angles and the CP-violating phase $\delta$ of the CKM matrix which tells us how quarks of different flavor mix with each other},
%\item{Nine yukawa couplings $y_i$ (i = 1 to 9) which determine the mass of nine charged fermions}.
%\end{itemize}


The heavy particles ($W^{\pm}$, $Z^0$, $H$, and top) can only be produced at high energy particle colliders like the Large Hadron Collider (LHC) operating at a center-of-mass energy of 13 TeV in Geneva, Switzerland.
Until the 90's the existence of almost all the particles of the SM were confirmed except the top quark and the Higgs boson. 
These had eluded previous experiments due to difficulties in the production or reconstruction of its decays.
The top quark was discovered in 1995 in the Tevatron collider of the Fermilab laboratory, this proton collider operated with a center of mass energy of 1.8 TeV untill 2010.
%The LHC collider at the CERN laboratory in Geneva, Switzerland, began its operations in 2010 colliding protons at 7 TeV increasing the colliding energies to 8 and 13 TeV in the subsequent years.
In 2012, the ATLAS and CMS experiments, with detectors at two points where the protons collide in the LHC, announced the discovery of a new boson with a mass of 125 GeV.
This particle has been identified as the Higgs boson of the SM by measuring its properties and comparing to those predicted by the SM.
%So far, all measurements of the properties of this boson are consistent with those of the Higgs boson of the Standard Model (SM).


The Compact Muon Solenoid (CMS) experiment is located at one of the four interaction points of the LHC.
%It is designed to detect particles known as muons very accurately.
The CMS detector has the form of a cylindrical onion, with several concentric layers of components.
A powerful magnet is used to bend charged particles as they move away from the point of collision to identify the charge of the particles to bend them in opposite directions and measure momentum.
A silicon tracker, made of about 75 million electronic sensors arranged in concentric layers, measures the curvature of charged particles with very high precision \cite{Chatrchyan:2008aa}.
The electromagnetic calorimeter detects photons and electrons while the hadronic calorimeter detects mainly pions and kaons.
The muons are detected by special chambers placed outside the solenoid as shown in Figure~\ref{figure5}.


During Run 1 (2011 and 2012) the LHC reached a peak instantaneous luminosity of 7.7 $\times$ $10^{33}$ $cm^{-2}s^{-1}$ and delivered an integrated luminosity of about 25 $fb^{-1}$ to each ATLAS and CMS.
%During Run 2 (2015-2018), the total delivered luminosity was about 150 $fb^{-1}$, but the precision on this value remains to be determined.
%The LHC will deliver about 300 $fb^{-1}$ by 2024 \cite{collaborations2019report}.
The goal of subsequent runs of the LHC has been to obtain datasets at higher values of luminosities for precision measurements of the properties of the Higgs boson and other SM particles.
%, in order to test the Standard Model pattern of couplings to elementary particles.
%In order to minimize the machine downtimes and maximize the productive use of the LHC for physics, the replacement of the inner triplet magnets (the one responsible to squeeze the beam at collision) and  of  all  hardware  changes  needed  to  enable  an  ambitious  luminosity  upgrade  will  take  place in parallel during one shutdown, at around 2023-25 (LS3), with some of the modification anticipated in 2019-2020 (LS2).
For the final phase of the LHC, the  High Luminosity LHC (HL-LHC), the goal is to obtain a total of 3000 $fb^{-1}$ by the year 2036 as shown in Figure~\ref{figure6}.
%Figure~\ref{figureKappas} shows a summary of the current measurements of the Higgs coupling parameters with the LHC Run 1 dataset (2011,2012)  using ggF, VBF, VH, and ${t\bar{t}H}$ production modes \cite{Tanabashi:2018oca}.
%Figure~\ref{figureKappas} shows a summary of the current measurements of the Higgs coupling parameters with the LHC Run 1 dataset (2011,2012)  using ggF, VBF, VH, and ${t\bar{t}H}$ production modes \cite{Tanabashi:2018oca}.
Figure~\ref{figureKappasUncs} shows the precision expected for the Higgs coupling parameters  with 3000 $fb^{-1}$, and shows that one of the dominant uncertainties remaining is due to the luminosity measurement.

During Run 2 (2015-2018), the total delivered luminosity was about 150 $fb^{-1}$, but the precise value and its uncertainty remains to be determined and is the subject of the project outlined in this document.



%\begin{figure}[H]
%        \centering
%        \includegraphics[width=\columnwidth]{./pg.png}
%        \caption{Important production processes of Standard model Higgs boson ggF, VBF, VH, ttH and tH in proton collisions.}
%        \label{figure 2}
%\end{figure}


%\begin{figure}[H]
%  \centering
%   \includegraphics[width=\columnwidth]{./cd2.png}
%  \caption{Standard model Higgs boson production cross section with center of mass energy and branching ratios for various decay channels.}
%   \label{figure 3}
%\end{figure}

%\begin{figure}[H]
%  \centering
%  \includegraphics[scale=0.4]{./couplings.png}
%  \caption{ATLAS-CMS combined measurements of coupling strength modifiers \cite{Tanabashi:2018oca}.}
%  \label{figureKappas}
%\end{figure}


\begin{figure}[H]
  \centering
  \includegraphics[width=0.7\columnwidth]{./LHCcomplex.png}
  \caption{Diagram of the LHC accelerator complex located in Geneva, Switzerland. Four interaction points are shown for the ALICE, ATLAS, LHCb, and CMS experiments.}
  \label{figure5}
\end{figure}

\begin{figure}[H]
  \centering
  \includegraphics[width=0.7\columnwidth]{./cms12.png}
  \caption{Transverse view of the CMS detector showing the silicon tracker, electromagnetic calorimeter, hadron calorimeter, superconducting solenoid and muon chambers \cite{Chatrchyan:2008aa}.}
  \label{figure5}
\end{figure}

\begin{figure}[H]
  \centering
  \includegraphics[width=0.8\columnwidth]{./HLLHCLumi.png}
  \caption{Projected performance of the LHC until 2038, which shows the preliminary dates for prolonged stops (LS) of the LHC and luminosities. Points show instantaneous luminosity while the line shows luminosity accumulated \cite{collaborations2019report}.}
  \label{figure6}
\end{figure}


 \begin{figure}[H]
   \centering
   \includegraphics[width=0.5\columnwidth]{./higgs_couplings.png}
   \caption{Expected uncertainties on the Higgs coupling strength modifiers with 3000 $fb^{-1}$ of proton-proton collision data.}
   \label{figureKappasUncs}
 \end{figure}



%\newpage
\section{PROPOSAL}

\onehalfspacing
In this project, it is proposed to


\section{GENERAL OBJECTIVE}

%- motivation, tH production crossection, sign of $k_t$, search for BSM, ..

\onehalfspacing
There is currently a great enigma in physics since we know that ordinary matter comprises only $5\%$ of the universe, another $27\%$ includes Dark matter and the rest $68\%$ is Dark energy.


\section{HYPOTHESIS}
%%  - what we expect to find

\onehalfspacing

%\newpage
\section{SPECIFIC OBJECTIVES}

\onehalfspacing

In order to complete a journal publication ...

\begin{itemize}
\item Simulation..
\end{itemize}

Finally, the results of this project will be presented at national or international conferences and a publication in a scientific journal will be accomplished.


\section{METHODOLOGY}

\onehalfspacing
The methods for ...


\textsc{Root} is a \textsc{c++} based software that allows visualization of event distributions and variable correlations.


\section{EXPECTED RESULTS}
%- a limit, also predictions for future runs

\onehalfspacing The results from this project include the following items:
\begin{itemize}
\item The student will become part of an international collaboration.
\item A leading contribution ....  resulting in a publication in a scientific journal.
\item Participation in the CMS detector Phase II upgrade through development and possible installation of particle detectors possibly resulting in a technical report.
\item A presentation of a poster or a talk in one of the national conferences of the Division of Paticles and Fields of the Mexican Society of Physics or a presentation of a poster or a talk in an international physics conference.
\item An academic stay at a scientific center like CERN or Fermilab.
\end{itemize}



\section{CALENDAR OF ACTIVITIES}
\onehalfspacing
\begin{itemize}

\item {\bf Semester 1 (2019-2)}:
  \SubItem{ Readings on Standard Model theory}
  \SubItem{ Basic Linux computing skills (\textsc{Bash, Emacs, Root})}
  \SubItem{ Initial planning of the analysis strategy}

\item {\bf Semester 2 (2020-1)}:
  \SubItem{ Course I on particle physics and/or particle detection}
  \SubItem{ Basic Linux computing skills (\textsc{Bash, Emacs, Root})}
  \SubItem{ Readings on SM and tH literature}
  \SubItem{ Computing accounts at CERN and Fermilab}
  \SubItem{ CMS service work for authorship}
  \SubItem{ Possible summer stay at CERN or Fermilab}

\item {\bf Semester 3 (2020-2)}:
  \SubItem{ Course II on particle physics and/or particle detection.}
  \SubItem{ CMS service work for authorship}

\item {\bf Semester 4 (2021-1)}:
  \SubItem{ CMS service work for authorship}

\item {\bf Semester 5 (2021-2)}:
  \SubItem{ Presentation of poster or talk at national or international conference}

\item {\bf Semester 6 (2022-1)}:
  \SubItem{ Review process of the results within the CMS collaboration}
  \SubItem{ Presentation of poster or talk at national or international conference}
  \SubItem{ Writing of the paper publication}
  \SubItem{ Writing of the thesis}
\end{itemize}



\cleardoublepage
\onehalfspacing
\bibliographystyle{unsrt}
\bibliography{paper}

\end{document}

