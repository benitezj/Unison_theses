\chapter{Summary and Outlook}

The main objective of this work was the calibration of the luminosity measured using the PCC detector through the van der Meer (vdM) method, contributing to the publication, which focuses on the second part of Run 2, covering the years 2017 and 2018. Throughout the process, the PCC results were developed in parallel with those from other CMS luminometers, ultimately achieving a final uncertainty below the 1\% target required for publication. The work addressed key topics such as the importance of luminosity, the LHC and the CMS experiment, the PCC method and the calibration procedure for estimating $\sigma_{\text{vis}}$, and the treatment of all systematic uncertainties.

This study provided a comprehensive overview of how Run 2 data were collected and processed. For each year, the vdM calibration program is described, detailing the full processing chain for the PCC detector, from raw data files with cluster reconstruction to the selection of "good" pixel detector modules. The study also includes the afterglow corrections to the physics data to achieve the most accurate luminosity possible. For the calibration the optimal fit model selected was a Poly4G function, using a background substraction per bunch for 2017 and a constant value for 2018, based on the analysis of Super Separation periods.

The PCC detector showed good performance compared to the other CMS luminometers. After all corrections, the final results for $\sigma_{\text{vis}}$ were \textbf{4666.4 $\pm$ 14.7 mb} for 2017 and \textbf{964.9 $\pm$ 2.2 mb} for 2018. The variations remained within acceptable ranges across different comparisons, such as scan-to-scan and bunch-by-bunch during the calibration procedure. Moreover, the PCC demonstrated excellent stability and linearity in the integrated luminosity throughout the year, contributing directly to the overall analysis and to the final uncertainties reported in the publication: \textbf{0.86\%} for 2017 and \textbf{0.83\%} for 2018.




