\chapter{Conclusiones}

En esta tesis se ha llevado a cabo un análisis exhaustivo del modelo de ruido para la medición de luminosidad utilizando el método de conteo de clusters de píxeles (PCC) en el experimento CMS. A través de este estudio, se ha demostrado que el método PCC es una herramienta robusta y precisa para medir la luminosidad, gracias a su alta granularidad y baja ocupación en el detector de píxeles de silicio. La observación y caracterización del afterglow, tanto de tipo 1 como de tipo 2, ha permitido comprender mejor los efectos residuales que influyen en las mediciones de luminosidad, lo que es crucial para garantizar la precisión de los datos obtenidos en el LHC.

El modelo matemático propuesto, que incluye contribuciones del afterglow y del ruido residual, ha mostrado un buen ajuste a los datos experimentales, como lo evidencian los residuos distribuidos aleatoriamente alrededor de cero. Además, el análisis de la evolución temporal de los parámetros del modelo ha revelado que estos presentan un comportamiento estable a lo largo del tiempo, lo que refuerza la validez del modelo utilizado.

En resumen, este trabajo contribuye a mejorar la precisión de las mediciones de luminosidad en el experimento CMS, lo que es fundamental para la búsqueda de nuevas partículas, la medición de las propiedades de partículas conocidas y la detección de procesos raros. Los resultados obtenidos abren nuevas perspectivas para futuros estudios en física de partículas, especialmente en lo que respecta a la optimización de los métodos de medición de luminosidad en colisionadores de alta energía.
