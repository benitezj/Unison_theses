\chapter{Resumen}
En este trabajo se presenta un estudio detallado del modelo de ruido para la medición de luminosidad utilizando el método de conteo de clusters de píxeles (Pixel Cluster Counting, PCC) en el experimento CMS del Gran Colisionador de Hadrones (LHC). La luminosidad es un parámetro fundamental en los experimentos de física de partículas, ya que permite cuantificar la tasa de colisiones y, por ende, la producción de eventos de interés. El método PCC aprovecha la alta granularidad del detector de píxeles de silicio del CMS para medir la luminosidad con una excelente linealidad respecto al número de interacciones por cruce de haces (pile-up). 

Se analizan datos adquiridos durante el año 2022, con un enfoque particular en la observación del fenómeno de afterglow, que consiste en señales residuales que afectan las mediciones posteriores a los bunches principales. Se propone un modelo matemático que describe tanto el afterglow tipo 1, de naturaleza electrónica, como el afterglow tipo 2, generado por partículas secundarias provenientes de la activación del material circundante. Además, se estudia la evolución temporal de los parámetros del modelo, lo que permite evaluar la estabilidad y consistencia del mismo. Los resultados obtenidos demuestran la capacidad del método PCC para proporcionar mediciones precisas de luminosidad, contribuyendo así a una mejor comprensión de los efectos que influyen en la detección de partículas en el LHC.

