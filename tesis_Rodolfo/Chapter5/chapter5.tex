\chapter{Summary and Outlook}
This work focuses on the luminosity calibration for the PCC CMS luminometer from the van der Meer scans performed by the CMS Collaboration in summer 2018 for proton-proton collisions at $\sqrt{s}=$ 13 TeV. Some aspects such as the luminosity and its importance, the CMS experiment, the PCC luminometer, and the calibration procedure to estimate $\sigma_{vis}$ were covered.

A key part of this work was the reprocessing of the recorded data. One part of the reprocessing was applying a different selection of the Pixel detector modules, based on removing instabilities in luminosity measurement for the corresponding calibration period. The other part of the reprocessing concerns to the data process work flow, where a new step was added, exploiting the full statistics of the data, providing an improvement in the statistical uncertainty in the rates by a factor of 10, in the fit model and a better determination of the beam parameters, relative to the previous analysis reported. This new data implied the estimation of the background correction to the rates (PCC) from the two super separation periods, being this estimation also part of the analysis of this work.

To determine $\sigma_{vis}$, seven scan pairs (four standar vdM and three Imaging scans) were used. A systematic variation is observed from bunch to bunch and scan to scan, being the first vdM scan lower than the rest of the scans. The assinged value of the calibration is $\sigma_{vis}= 9229 \pm 8(stat.)\pm 28 (syst.) \text{ mb}$.

The analysis and results reported in this work will be useful for an analysis of the precise luminosity measurement for Run2 in the CMS experiment of LHC, where the rest of the luminometers and its uncertainties will be studied (and improved).






%$$RMS=\sqrt{\frac{1}{N} \sum^{N}_{i}(x_{i}-\mu)^{2} } $$
%$$RMS=\sqrt{\frac{1}{N} \sum^{N}_{i}(x_{i})^{2} } $$
%$$Syst_{\text{unc}}=RMS-Stat_{\text{unc}} $$

