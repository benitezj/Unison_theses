%\documentclass[final,3p]{article}
\documentclass[final,12p]{article}
\usepackage{graphicx}
\usepackage[letterpaper,margin=1.0in]{geometry}
\usepackage{amssymb}
\usepackage{changepage}
\usepackage{float}
\usepackage{hyperref}
\usepackage{url}
\usepackage{afterpage}
\usepackage{natbib}
\usepackage{setspace}
\usepackage{fancyhdr}
\pagestyle{fancy}
\fancyhf{}
\usepackage[utf8]{inputenc} 
\usepackage{lastpage}


\def\Student{Julio C\'{e}sar Borb\'{o}n Fragoso}
\def\Title{ANTEPROYECTO}
\def\Prog{Maestr\'{i}a en Ciencias (F\'{i}sica) }
\def\Dept{Departamento de Investigac\'{i}on en Fis\'{i}ca}
\def\Institution{Universidad de Sonora}
\def\Director{Dr. Jos\'{e} Feliciano Ben\'{i}tez Rubio}
\def\ProjectTitle{Luminosity measurement using Pixel Cluster Counting in the CMS experiment}
\def\ResearchLine{Experimental Particle Physics}

%%header and footer
\lhead{\Student\ / \Prog}
\rhead{\Title}
\rfoot{Page \thepage \hspace{1pt} of \pageref{LastPage}}


%%%%%%%comands
\newcommand{\SubItem}[1]{ {\setlength\itemindent{15pt} \item[-] #1} }
\newcommand{\lumi}[1]{{#1~fb$^{-1}$}}
\newcommand{\instlumi}[1]{#1$\times 10^{34}$ cm$^{-2}$s$^{-1}$}
  

%%%%%%%%%%%%%%%%%%%%%%%%%%%%%%%%%%%%%
\begin{document}
\onehalfspacing

%%%%Title Page
\begin{titlepage}
\centering
\hspace{0pt}
\vfill
{\scshape\Large \Title \par}
%project revised 2023-2
  
  \vspace{2cm}
  {
    TITLE:\par
    {\bf \large \ProjectTitle \par}
  }
       
  \vspace{0.5cm}
  {
    RESEARCH LINE: \par
    \ResearchLine \par
  }
        
  \vspace{4cm}
  {\underline{\hspace{8cm}}\par}
  {\bf \scshape \Student \par}
  {Student\par}

  \vspace{1cm}
  {\underline{\hspace{8cm}}\par}
  {\scshape \Director \par}
  {Director\par}

  \vspace{1cm}
  {\bf \Prog \par}
  {\Dept \par}
  {\Institution \par}

  \vspace{4cm}
  {\today}

\hspace{0pt}
\vfill

\end{titlepage}


%%%%% white page for print out
\shipout\null


%%%% Abstract Page
\newpage
\hspace{2pt}
\vfill

  \begin{center}
    {\Large \ProjectTitle \par}
    \vspace{1cm}
    {\itshape\textbf{Abstract}\par}
  \end{center}
  
  \vspace{0.5cm}
   
This research project proposes to measure the luminosity from RUN 3 made in 2024 with the CMS data using mainly the PCC method. Accurate luminosity measurements are crucial for understanding particle physics, including Higgs boson production. The research involves analyzing the CMS pixel detector, calibrating it for 13.6 TeV collisions, and addressing systematic uncertainties.

  \hspace{2pt}
\vfill



%%%%%% Begin the body
\newpage
\section{BACKGROUND}


The standard model (SM) of particle physics is so far the best theoretical model to describe the interaction of elementary particles mediated by three of the four fundamental forces of nature which are electromagnetic force, strong nuclear force and the weak nuclear force. The SM is divided into two categories, the bosonic sector and the fermionic sector.
The bosonic sector contains particles which mediate the fundamental forces of nature and the fermionic sector contains particles which make up all known matter in our universe.
There are three generations of fermion particles: the first generation consists of up (u) quark, down (d) quark, electron and electron neutrino, the second generation consist of charm (c) quark, strange (s) quark, muon and muon neutrino, and the third generation has the top (t) quark, bottom (b) quark, tau and tau neutrino.
The bosonic sector consists of the gauge bosons: gluon, photon, $W^{\pm}$, $Z^0$ which mediate strong nuclear force, electromagnetic force and weak nuclear force respectively.
The Higgs boson ($H$), is the last of the gauge bosons, it gives mass to the other SM particles via electroweak symmetry breaking mechanism \cite{Chatrchyan:2012xdj}.
The heavy particles ($W^{\pm}$, $Z^0$, $H$, and top) can only be produced at high energy particle colliders like the Large Hadron Collider (LHC) operating at a center-of-mass energy of 13 TeV in Geneva, Switzerland.

\begin{figure}[H]
  \centering
  \includegraphics[width=0.6\columnwidth]{./sm1.png}
  \caption{Particles of the Standard Model
  
  }
  \label{figure1}
\end{figure}




The large hadron collider (LHC) is the world largest and more powerful particle accelerator. It first started in 10 september of 2008 and remains the latest addition to CERN's accelerator complex. The LHC consist of a 27 kilometer ring of superconducting magnets with a number of accelerating structures to boost the energy of the particles along the way. All the controls for the accelerator, its services and technical infrastructure are housed under one roof at the CERN Control Centre. From here, the beams inside the LHC are made to collide at four locations around the accelerator ring, corresponding to the positions of four particle detectors; Atlas, CMS, Alice and LHCb.

Until the 90s, existence of almost all the SM particles were confirmed except the top quark and the Higgs boson. 
These had eluded previous experiments due to difficulties in the production and reconstruction of its decay products.
The top quark was discovered in 1995 at the Tevatron collider of the Fermilab laboratory, this proton collider operated with a center-of-mass energy of 1.8 TeV until 2010.
In 2012, the ATLAS and CMS experiments, with detectors placed at two points where the proton beams collide in the LHC, announced the discovery of a new particle with a mass of 125 GeV.
This particle has been identified as the Higgs boson by measuring its properties and comparing to those predicted by the SM.


Luminosity, $L$, is a key parameter at particle colliders along with the energy available in the collision.
$L$ is one of the  main figures of merit that quantify the potential for observing new particles and measuring their properties.
The instantaneous luminosity $L(t)$ is the process-independent ratio between the rate $R(t)$ of events produced per unit time and the cross section for a given process $\sigma$:  $R(t) = \sigma L(t)$.

In order to evaluate the total number of events we used integrated luminosity, it is a measurement of the collected data size, and it is important value to characterize the performance of an accelerator, defined as: \\ 


\begin{center}
$L_{inst} = \int _{0}^{t} L_{inst}(t') dt'$ \\ 
\end{center}


Because it directly relates to the number of observed events; $L_{int} \dot \sigma_{p}$ = number of events of interest. \\

\begin{figure}[H]
  \centering
  \includegraphics[width=0.7\columnwidth]{./integratedlum.png}
  \caption{
   Total integrated luminosity from previous data from CMS
  }
  \label{Integrated luminosity}
\end{figure}

During Run 1 (2011-2012) LHC reached a peak instantaneous luminosity of \instlumi{0.77} and delivered an integrated luminosity of about \lumi{25} with a precision of about 2.0\% 
\footnote{1 barn is a unit of area corresponding to $10^{-24}$ cm${^2}$ and 1 femtobarn (fb) = $10^{-39}$ cm$^{2}$. For comparison, the total Higgs production cross section is 48600 fb.}.


In the first part of Run 2 (2015-2016), the delivered luminosity has been measured to be \lumi{38.4} with an unprecedented precision of 1.3\% \cite{Sirunyan:2021qkt}.
For the second part of Run 2 (2017-2018), the integrated luminosity is about \lumi{78}, but its precise value and uncertainty remain to be determined \cite{CMS:2018elu}.
The plan of the LHC till year 2038 is to obtain datasets with up to 10 times higher values of instanteneous luminosities in the final phase.
The LHC Run 3 began in 2022 and will last until 2024, with an expected integrated luminosity of about \lumi{450} \cite{lumi-run3}. 



 
\begin{figure}[H]
  \centering
  \includegraphics[width=0.6\columnwidth]{./LHCcomplex.png}
  \caption{
    Diagram of the LHC accelerator complex located near Geneva, Switzerland. The complex consists of three accelerator stages: the proton (p) or lead (Pb) source, the Proton Synchrotron (PS), the Super Proton Synchrotron (SPS), and the 27 km LHC ring. Four collision points are shown corresponding to the ALICE, ATLAS, LHCb, and CMS detectors  \cite{Mobs:2684277}.
  }
  \label{figure2}
\end{figure}


\begin{figure}[H]
  \centering
  \includegraphics[width=0.8\columnwidth]{./HLLHCLumi.png}
  \caption{
    Projected performance of the LHC until 2038, which shows the preliminary dates for prolonged stops (LS) of the LHC and luminosities. Points show instantaneous luminosity while the line shows luminosity accumulated \cite{collaborations2019report}.
  }
  \label{figure6}
\end{figure}



\section{PROPOSAL}

In this project, we aim to make the measurements of the luminosity for the preliminary data from Run 3 take on 2024 using the Pixel Cluster Counting (PCC) method. 

\section{GENERAL OBJECTIVE}


The main objective of this work is to obtain calibration data for 2024 aiming for a uncertainty less than 2\% in this Run. This results are important for CMS to compare physical process with the Standard Model, these results are likely to be presented on conferences and reviews on 2025.
This precision is needed for the measurement of many important physics processes, like the Higgs boson production crossection, which define the Standard Model of particle physics. 

\section{HYPOTHESIS}


Based on the previous work on the Run 2 (2015-2016) datasets which have achieved a uncertainty of  2.3\% \cite{Collaboration:121}, we expect that a precision at 2\% will be possible due to the improved methods we are developing.
For reference, the current systematic uncertainty on the Run 2 2017 was 2.3\% \cite{Collaboration:122} for 2018 was 2.5\% \cite{Collaboration:123} and for Run 3 (2022) was 1.4\% \cite{Collaboration:124}. 



\section{METHODOLOGY}


The CMS experiment is located at one of the four interaction points of the LHC.
The CMS detector has the form of a cylindrical onion, with several concentric layers of components.
A powerful magnet is used to bend charged particles as they move away from the point of collision to identify the charge and measure their momentum.
A silicon tracker, made of about 75 million electronic sensors arranged in concentric layers, measures the curvature of charged particles with very high precision \cite{Chatrchyan:2008aa}.
The electromagnetic calorimeter detects photons and electrons while the hadron calorimeter detects mainly pions and kaons.
The muons are detected by special chambers placed outside the solenoid as shown in Figure~\ref{fig:CMS}.

The PCC method for measuring luminosity uses the Pixel detector of the CMS tracker, the layout of the  detector used for recording the data during 2016-2018 is shown in Figure~\ref{fig:pixeldet},  this detector is expected to deliver high-quality data until the end of LHC Run 3 (currently expected for 2025).
The Pixel detector consists of 4 concentric cylindrical layers in the barrel and 3 disks in each endcap.
Each detector part is composed of pixel sensors, a schematic of one sensor is shown in Figure~\ref{fig:pixeldet}.
The entire Pixel detector contains 1856 sensor modules and a total of 124 million pixels \cite{TrackerGroupoftheCMS:2020bgg}.

The PCC method consists of the reconstruction of track clusters produced by charged particle tracks as shown in Figure~\ref{fig:bunchcrossing}.
Due to the fine granularity of the pixel sensors and the large number of total pixels, the hit occupancy in the sensors remains very small, order of 1\%, during normal collisions.
This low occupancy makes the PCC  very linear as a function of pileup, an essential property of a good luminometer \cite{Sirunyan:2021qkt}.


The calibration of the luminometer consists of a van der Meer (vdM) scan performed in a special LHC run usually at the beginning of the run period (year).
These scans are performed by varying the separation between the beams in each direction ($x$ and $y$) at a fixed number of separation steps of about 100 micrometers. Then these rates are fitted by a Gaussian function to obtain the beam overlaps $\Sigma_{x,y}$ and the peak rate of clusters $R(0,0)$ \cite{CMS:2018}. A schematic view of this method is shown in Fig. \ref{vdMSketch}. 
Finally the calibration constant ($\sigma_{vis}$) is determined as Eq.\ref{sigmavis_eq}.

\begin{equation}
  \sigma_{vis}=\frac{R_{det}}{L_{b}}=\frac{2\pi \Sigma_{x} \Sigma_{y} R(0, 0)}{N_{1}N_{2} f}
  \label{sigmavis_eq}
\end{equation}
Where $f$  is the LHC frecuency and $N_{1},N_{2}$ are the buch current used to normalize  the rates.
This calibration constant is then used to determine the luminosity during a normal running throughout the data-taking year \cite{CMS:2018}.

%It is expected that coincidences will be less sensitive to backgrounds in HL-LHC pileup conditions and therefore a more robust luminosity measurement can be performed.


\begin{figure}[H]
  \centering
  \includegraphics[width=0.65\columnwidth]{./cms12.png}
  \caption{Transverse view of the CMS detector showing the silicon tracker, electromagnetic calorimeter, hadron calorimeter, superconducting solenoid and muon chambers \cite{Chatrchyan:2008aa}.}
  \label{fig:CMS}
\end{figure}


\begin{figure}[H]
  \centering
  \includegraphics[width=0.7\columnwidth]{./PixelDetectorPhase1.png}
  \includegraphics[width=0.27\columnwidth]{./PixelSensor.png}
  \caption{
    Left: diagram showing the layout of the CMS Pixel detector used during 2016-2018.
    The layout consists of 4 barrel layers and 2 endcap disks with two rings each.
    Right: a diagram showing the structure of one pixel sensor.
    The entire detector consists of 1856 sensors and 65 million pixels \cite{TrackerGroupoftheCMS:2020bgg}.
  }
  \label{fig:pixeldet}
\end{figure}

\begin{figure}[H]
  \centering
  \includegraphics[width=0.48\columnwidth]{./bunchcrossing.jpg}
  \includegraphics[width=0.42\columnwidth]{./vectorhit1.jpg}
  \caption{
    Left: Dagram showing the collision of two proton bunches at LHC, bunches contain about $10^{11}$ protons  \cite{deMaria:2008zzb}.
    Right: diagram showing example tracks originating from the collision and producing hits in the pixel detector layers \cite{Thomsom}.
  }
  \label{fig:bunchcrossing}
\end{figure}


\begin{figure}[H]
  \centering
  \includegraphics[width=0.7\columnwidth]{./vdm_sketch.png}
  \caption{
   Sketch of a vdM scan in x and y planes. The sketch is an example of
the fit of the resulting cluster rates \cite{vdMSketch}.
  }
  \label{vdMSketch}
\end{figure}


\section{SPECIFIC OBJECTIVES}

For Run 3 2024 luminosity measurements:
\begin{itemize}
\item Understand the CMS pixel detector layout including the barrel layers and endcap disks and their constituent modules.
\item Study the evolution of the parameters of proton beams
\item Calculate the calibration constant $\sigma_{vis}$ corresponding to 13.6 TeV proton-proton collision data.
\item Study the variation of $\sigma_{vis}$ in function of the scan with time
\item Determine the systematic uncertainties on the calibration constant.
\item Determine the systematic uncertainties on the integrated luminosity.
\item Study the stability and linearity of the PCC luminosity measurement by comparing to other CMS luminometers.
\item Determine the instantaneous and the total integrated luminosity.
\end{itemize}

\par



\newpage
\section{EXPECTED RESULTS}

From this project the following is expected:
\begin{itemize}
\item Calibration of the PCC luminometer and the corresponding uncertainty.
\iten Preliminary measurements of the 2024 integrated luminosity for CMS
\item From 2024 data compute the uncertainty from the calibration to contribute with the luminosity measurement in 2024.  
\item A presentation of the results at a national conference. 
\end{itemize}

\section{Semestral calendary of the project}

\begin{figure}[H]
  \centering
  \includegraphics[width=0.7\columnwidth]{./diagram.jpeg}
  \caption{
   Diagram of this project
  }
  \label{Grantt Diagram}
\end{figure}

\section{Abilities to be develop by the student}
\begin{itemize}
\item Develop software abilities
\item Learn programation language  C, C++, Python.
\item Learn how to use Compute Cluster, Latex, Root, etc.
\item Data analysis 
\item Statistical analysis
\item Review publication
\end{itemize}


\section{Mobility actions}
We expect to present the results on a congress the next year 2025. 



%\begin{itemize}
%
%\item {\bf Semester 1 (2023-2)}:
%  \SubItem{ Readings on Standard Model of physics theory, LHC, and CMS experiments.}
%  \SubItem{ Review of the literature.}
%  \SubItem{ Master Linux computing and data analysis  skills (\textsc{Bash, Emacs, Root, HTCondor parallel processing.})}
%  \SubItem{ Complete work on the 2022 luminosity measurement for a preliminary  publication.}
%
%\item {\bf Semester 2 (2024-1)}:
%  \SubItem{ Course I on particle physics, particle detection, or data analysis.}
%  \SubItem{ Continue readings on Standard Model of physics theory, LHC, and CMS experiments.}
%  \SubItem{ Master Linux computing and data analysis  skills (\textsc{Bash, Emacs, Root, HTCondor parallel processing.})}
%  \SubItem{ Complete work on the Run 2 luminosity measurement for a final  publication.}
%  \SubItem{ Start work on the 2023 dataset luminosity measurement. }
%  
%\item {\bf Semester 3 (2024-2)}:
%  \SubItem{ Course II on particle physics, particle detection, or  data analysis.}
%  \SubItem{ Complete work on the 2023 dataset luminosity measurement.}
%  \SubItem{ Start work on the 2024 dataset luminosity measurement. }
%  
%\item {\bf Semester 4 (2025-1)}:
%  \SubItem{ Continue work on the 2024 dataset luminosity measurement. }
%  \SubItem{ Start studies on the TEPX luminometer for the HL-LHC.}
%  \SubItem{ Writing of the monograph.}
%
%\item {\bf Semester 5 (2025-2)}:
%  \SubItem{ Complete work on the 2024 dataset luminosity measurement. }
%  \SubItem{ Complete studies on the TEPX luminometer for the HL-LHC.}
%  \SubItem{ Presentation at national or international conferences of the TEPX upgrade studies.}
%  
%\item {\bf Semester 6 (2026-1)}:
%  \SubItem{ Start final studies of the Run 3 data luminosity measurement for publication in a peer reviewed journal.}
%
%\item {\bf Semester 7 (2026-2)}:
%  \SubItem{ Completion of the studies of the Run 3 data luminosity measurement for publication in a peer reviewed journal.}
%  \SubItem{ Presentations at national or international conferences of the luminosity measurements and TEPX luminometer for HL-LHC.}
%  \SubItem{ Writing of the Ph.D thesis.}
%  
%\item {\bf Semester 8 (2027-1)}: 
%  \SubItem{ Publication of the Run 3 data luminosity measurement in a peer reviewed journal.}
%  \SubItem{ Presentations at national or international conferences of the luminosity measurements and TEPX luminometer for HL-LHC.}
%  \SubItem{ Defense of the Ph.D thesis.}
%  
%\end{itemize}


\newpage
\onehalfspacing
\bibliographystyle{unsrt}
\bibliography{paper}

\end{document}

