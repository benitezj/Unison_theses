\begin{abstract}
    \begin{center}
    {\LARGE Desarrollo y calibración del sistema de medici\'on de luminosidad CMS Silicon Pixel Detector}
\end{center}

Se describe la calibraci\'on del lumin\'ometro de conteo de píxeles del CMS realizada mediante el programa de escaneo "Van der Meer" para el experimento CMS en la toma de datos en colisiones protón-protón a $\sqrt{s}=$13.6 TeV en el 2022. El período de recopilación de datos fue el 10 y 11 de noviembre para el fill 8381 del LHC, los datos se procesaron  aplicando una selecci\'on de  m\'odulos "buenos" en el  detector, basandose en un estudio de estabilidad de \'estos. El análisis de los escaneos van der Meer con algunas correcciones específicas arroja en la constante de calibraci\'on $\sigma_{vis}$ un valor de:   $4163.3 \pm 3.3 (\text{stat.}) \pm 13 (\text{syst.})\text{ mb}$. 

%Se describe la calibraci\'on de la medici\'on de luminosidad para la corrida 3 del gran colisionador de hadrones (LHC),  realizada mediante el programa de escaneo  "Van der Meer"  para el experimento CMS  durante 10 y 11 de Noviembre del 2022 . La toma de datos  proton-proton a $\sqrt{s}=$13.6 TeV fue con  el detector de pixeles de silicio (Silicon Pixel detector) para el  fill 8381, dichos datos se reprocesaron  aplicando una selecci\'on de  m\'odulos "buenos" en el  detector, basandose en un estudio de estabilidad de \'estos. El an\'alisis de 5 escaneos de Van der Meer presentados en este trabajo arroja en la constante de calibraci\'on $\sigma_{vis}$ un valor de:   $4163 \pm 3 (\text{stat.}) \pm 12 (\text{syst.})\text{ mb}$. 
\end{abstract}
