\begin{abstract}
\begin{center}
    {\huge Abstract}
\end{center}

The Pixel Cluster Counting (PCC) luminosity measurement method has emerged as a reliable offline technique for luminosity measurement developed for the CMS experiment.
%The PCC method is based on the rate of pixel clusters in the CMS pixel detector, which is the innermost part of the CMS tracker.
The PCC method is studied for its precision in luminosity measurement using 2017, 2018 and 2022  datasets. The results of these studies demonstrate that the PCC luminosity measurement method is stable and linear, able to achieve final uncertainty on the luminosity measurement at the level of 1\% or better.  Final systematic uncertainty achieved on luminosity measurement for 2022 %and 2022 datasets are 1.08\% and
is 1.5\%. %respectively. %1\% systematic uncertainty on the final luminosity measurement.
Moreover, the clusters and coincidences counting algorithms and their optimization for the Phase II High Luminosity LHC (HL-LHC) luminosity measurement are also developed, using simulated datasets for the proposed Phase II pixel detector. %This new pixel detector will replace the Run 2 pixel detector.
Linearity of the Tracker endcap pixel (TEPX) and Disk 4 Ring 1 luminometers are expected to be within 1\% (offline) up to a pileup of 200 and their statistical precision per second is about 0.1\% at pileup 200.
%It is expected to provide excellent statistical precision and linearity performance, which is essential for precision luminosity measurements at the HL-LHC. 
 
\end{abstract}
