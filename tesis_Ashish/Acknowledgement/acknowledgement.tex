I wish to express my profound gratitude to all those who have supported and guided me throughout my journey towards completing my PhD thesis in particle physics, with a focus on the CMS experiment at CERN. Their unwavering support, encouragement, and expertise have been indispensable to my success.

Foremost, I extend my deepest appreciation to my esteemed thesis supervisor, Dr. José Feliciano Benítez. His exceptional guidance, profound knowledge, and commitment to scientific exploration have profoundly shaped my research endeavors. Delving into the fascinating world of particle physics under his mentorship has been a privilege, and his continuous support has been pivotal at every stage of my thesis. I am immensely grateful for the opportunity to work under his supervision and for the personal and professional growth I have experienced through his invaluable guidance.

My heartfelt gratitude goes to my parents, Omkar Jagtaran Singh and Anita. Their unconditional love, encouragement, and sacrifices have been the bedrock of my academic journey. Their unwavering belief in my abilities and constant support have been my source of strength and motivation, enabling me to overcome challenges and pursue my passion for scientific exploration. Their constant presence and support throughout this demanding academic endeavor are cherished and deeply appreciated.

I also extend my appreciation to my sisters Priya, Monu, Sonu, Reena, and Seema. Their unwavering encouragement, understanding, and motivation have been an immense source of inspiration. Their belief in my abilities and continuous support have been a pillar of strength, urging me to remain focused and resilient even during challenging times.

I am thankful for the collective efforts of the scientific community at CERN and all contributors to the CMS experiment. Their collaboration, expertise, and dedication have fostered an enriching research environment, allowing me to partake in groundbreaking discoveries in particle physics.

I am indebted to my friends and colleagues, especially Rohit Gupta, for their valuable insights, stimulating discussions, and enduring support. Their encouragement, camaraderie, and intellectual exchanges have been instrumental in shaping both my research and personal growth.

To everyone mentioned, and those unnamed but equally appreciated, I extend my sincerest thanks. Your contributions, large and small, have been invaluable, and I am humbled by the generosity and support I have received. This thesis stands as a testament to your guidance, encouragement, and belief in my potential.

From the bottom of my heart, thank you.




























