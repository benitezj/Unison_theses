In this note the relative linearity between PCC and other luminometers has been studied using special runs recorded in 2018 which included pileup scans up to 100.
The linearity is determined by fitting the ratio of two luminometers as a funcion of SBIL measured by a reference detector, in this case the HFOC has been chosen as reference.
Isolated, leading, and train bunches are used to study the variations and train or afterglow effects.

Using isolated and leading bunches a slope of $0.32\pm0.04 \% [Hz/\mu b]^{-1}$ is obtained for the PCC/HFOC comparison.
For train bunches in fill 7358 a systematic decrease is observed in the slope as a function of the bunch id, where the slope becomes negative after 12 bunches.
This effect has been studied by removing the afterglow corrections for PCC and HFOC and by dividing the PCC into subdetectors.
This effect indicates the PCC suffers a loss in cluster counts possibly due to high occupancies at high pileup. 
A variation of the effect is observed for different parts of the Pixel detector.
