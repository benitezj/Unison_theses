\documentclass[final,11p]{CSP}
\usepackage{amssymb}
\usepackage{changepage}
\usepackage{float}
\usepackage{hyperref}
\usepackage{url}
\usepackage{afterpage}
\usepackage{natbib}
\usepackage{setspace}
\usepackage{fancyhdr}
\pagestyle{fancy}
\fancyhf{}

\def\Student{Ashish Sehrawat}
\def\Director{Dr. Jos\'{e} Feliciano Ben\'{i}tez Rubio}
\def\Title{MONOGRAPH}
\def\Prog{Doctorado en Ciencias (F\'{i}sica) }
\def\Dept{Departamento de Investigac\'{i}on en Fis\'{i}ca}
\def\Division{Division de Ciencias Exactas y Naturales}
\def\Universidad{Universidad de Sonora}

\def\ProjectTitle{Luminosity measurement with the Pixel detector at the CMS experiment of the Large Hadron Collider }
\def\ResearchLine{Astrof\'{i}sica, Cosmolog\'{i}a y F\'{i}sica de Part\'{i}culas}


\newcommand{\SubItem}[1]{
    {\setlength\itemindent{15pt} \item[-] #1}
}


%%header and footer
\lhead{\Student / \Prog }
\rhead{\Title}
\lfoot{\Dept}
\rfoot{Page \thepage}
\setlength{\headsep}{0.2in}
\renewcommand{\footrulewidth}{0.4pt}% default is 0pt


\begin{document}

%%%%Cover page
\begin{titlepage}
  \centering
  \hspace{0pt}
  \vfill
        {\scshape\Large \Title \par}

	\vspace{2cm}
        %\begin{adjustwidth}{2cm}{2cm}{
        %    TITLE:\par
            {\large \bf \ProjectTitle \par}
        %  }
        %\end{adjustwidth}

%	\vspace{0.5cm}
%        \begin{adjustwidth}{2cm}{2cm}{
%            RESEARCH LINE: \par
%            \ResearchLine \par}
%        \end{adjustwidth}

        
        \vspace{4cm}
        %{\underline{\hspace{8cm}}\par}
	{\scshape\large \Student \par}
        {PhD. Student\par}

        \vspace{1cm}
        %{\underline{\hspace{8cm}}\par}
	{\scshape \Director \par}
        {Thesis Director\par}

        \vspace{1cm}
        {\bf \Prog \par}
        {\Dept \par}
        {\Division \par}
        {\scshape \Universidad \par}

        \vspace{4cm}
	{\today}

\hspace{0pt}
\vfill

\end{titlepage}
%%%%% white page for print out
\shipout\null

%%%% Title and Abstract Page
\newpage
\hspace{11pt}
%\vfill

\begin{adjustwidth}{1cm}{1cm}

  \begin{center}
    {\Large \ProjectTitle \par}
    \vspace{0.5cm}

    {\Student \par}
    {\Universidad \par}  
    \vspace{1cm}
    
    {\itshape\textbf{Abstract}\par}
     \vspace{0.7 cm}
        
    \end{center}  

 
  \onehalfspacing The Pixel Cluster Counting (PCC) luminosity measurement method is an offline technique for the calculation of luminosity for any LHC run period based on counting the number of pixel clusters in the CMS pixel detector (innermost part of the CMS tracker) using minimum bias events. We investigate stability of PCC luminosity using 2018 CMS data for various subdetectors in the upgraded Phase I pixel detector. For Run 2 data, we aim to achieve a 1\% uncertainty on the final luminosity measurement. Also, we develop the algorithms and optimization for the Phase II High Luminosity LHC (HL-LHC) luminosity measurement using the Tracker Endcap Pixel Detector (TEPX) for which excellent statistical and linearity performance is expected.


    
\end{adjustwidth}

\hspace{2pt}
%\vfill
\vspace{1 cm}

%%%%%%%%%%%%%%%%%%%
\clearpage
\shipout\null


%%%%%% Begin the body
\newpage
%\onehalfspacing
\tableofcontents

\clearpage
\newpage
In Run III and HL-LHC phases, the LHC is expected to run at peak instantenous luminosities of about 2\ \instlumiunit\ and 5-7\ \instlumiunit, respectively \cite{hllhc}.
For the HL-LHC the proton bunch intensities will increase and will cause the pileup to reach peak values of 140-200, by comparison the Run II average pileup was about 34 with peak values around 50. 
During 2018, runs with special trigger configuration were recorded with pileup up to 100, these runs allow for studies of the effects of high pileup on the relative linearity between luminometers.
The luminosity measurement based on Pixel Cluster Counting (PCC) \cite{CMS-PAS-LUM-12-001,CMS-PAS-LUM-13-001,PCC2016_publicplots} method is expected to have a good linearity to high pileup due to its fine granularity.  


\section{The Large Hadron Collider}
\label{sec:lhc}
(describe the accelerator complex, proton bunches, c.m. energy, all experiments, run periods, inst. and integrated lumi, Run 3 and HL-LHC plans) \\

\onehalfspacing CERN accelerator complex is a progression of machines namely Linac 2, Proton Synchrotron Booster (PSB), Proton Synchrotron (PS), Super Proton Synchrotron (SPS) and Large Hadron Collider (LHC) as shown in Fig. 1 that accelerate particles mainly protons to increasingly higher energies of 50 MeV, 1.4 GeV, 25 GeV, 450 GeV and 14 TeV respectively \cite {CERNBrochure2017002Eng}. Each machine is used to increase the energy of particle before the beam is injected into to the next machine. Linear accelerator (Linac2) is the starting point for protons used in various experiments at CERN. After proton energy reaches 50 MeV, protons enter PSB which further increases its energy before they are injected into PS which takes up its energy to 25 GeV. After reaching 25 GeV, protons enter SPS which further increases its energy to 450 GeV. The protons are finally transferred to two beam pipes of LHC kept at very high vacuum.  \\ 

The Large Hadron Collider (LHC) is the most powerful particle accelerator in the world primarily build to collide protons and heavy ions at 7, 8, 13 and 14 TeV during different LHC run period. The LHC consists of a 27 kilometre ring of superconducting magnets producing strong magnetic field which is used to guide particle beams along the LHC ring. Proton beam made up of bunches travelling in opposite directions kept in two ultrahigh vacuum tubes are accelerated close to the speed of light using time dependent electric field produced by radiofrequency cavities before they collide at four interaction points along the LHC ring where ATLAS, ALICE, CMS and LHCb detectors are placed \cite {CERN-Brochure-2017-002-Eng}. \\

The first proton beam was successfully maneuvered around the LHC ring in September 2008 that marked the beginning of Run 1 with proton beams colliding at 7-8 TeV center-of-mass energy and this run period ended in 2013 \cite{cmsrun} \cite{Alemany-Fernandez:1631030}. Run 2 started in 2015 with protons colliding at 13 TeV center-of-mass energy and continued till October 2018 \cite{Wenninger:2668326}. Run 3 is expected to begin at the start of March 2022 and High Luminosity (HL)-LHC phase is expected to begin in mid-2027 \cite{Dainese:2019rgk}. Proton Synchroton (PS) is responsible for providing proton bunches spaced 7 metres (25 ns) apart with each bunch containing more than 100 billion protons. The maximum number of bunches reached with the beam preparation method used during Run 2 is 2556 \cite{Wenninger:2668326}. The number of protons in the colliding bunches, time of separation between the bunches and effective area of the bunch which depend on the rms transverse bunch size along horizontal and vertical directions is one of the ways to calculate instantaneous luminosity (L_{inst}). \\

Luminosity is a key performance parameter of an accelerator as it is a measure of number of collisions. Luminosity can be categorized into two types: instantaneous and integrated luminosity. Instantaneous luminosity is defined as the number of collisions occuring per second in per centimeter square area while integrated luminosity is obtained by integrating instantaneous luminosity over a time interval. Both are very important variables as they are used to obtain the number of collision events. During Run 1, peak value of LHC instantaneous luminosity was $0.77 \times 10^{34} \: cm^{-2} s^{-1} $ with $30 \: fb^{-1}$ integrated luminosity \cite{cmsrun1lumi}  . During Run 2, LHC instantaneous luminosity reached a peak value of $1.58 \times 10^{34}  \: cm^{-2} s^{-1}$ with $150 \: fb^{-1}$ integrated luminosity as shown in Fig. 2 \cite{CMS:2018elu}. At the end of Run 3, we expect to achieve an integrated luminosity  of $300 \:fb^{-1}$ while Phase 2 HL-LHC is expected to collect $3000 \:fb^{-1}$ of data as shown in Fig. 3 thereby increasing luminosity by a factor of 10 beyond the LHC's design value \cite{Dainese:2019rgk}. \\


\begin{figure}[H]
  \centering
  \includegraphics[width=0.8\columnwidth]{./LHCcomplex.png}
  \caption{\onehalfspacing Diagram of the LHC accelerator complex located in Geneva, Switzerland. Four interaction points are shown for the ALICE, ATLAS, LHCb and CMS experiments. \cite{Mobs:2684277}}
  \label{fig:LHC}
\end{figure}


\begin{figure}[H]
  \centering
  \includegraphics[width=1 \columnwidth]{./run1-2_lumi.png}
  \caption{\onehalfspacing Left: Instantaneous luminosities for Run 1 and Run 2 period. Right: Integrated luminosities for Run 1 and Run 2 period. \cite{lumi}}
  \label{fig:LHClumi}
\end{figure}


\begin{figure}[H]
  \centering
  \includegraphics[width=0.7 \columnwidth]{./lumi_projection.png}
  \caption{ \onehalfspacing Projected performance of the LHC until 2038, which shows the preliminary dates for prolonged stops (LS1, LS2, LS3, LS4, LS5) of the LHC and luminosities. Red points show instantaneous luminosity $(L_{inst})$ while the blue line shows integrated luminosity. \cite{collaborations2019report}.}
  \label{fig:LHCPlans}
\end{figure}


\section{The CMS experiment}
\label{sec:cms}
(describe the experiment in general, its discovery, mention the international collaboration)


\subsection{CMS detector}
(describe the detector components and what they do)


\begin{figure}[H]
  \centering
  \includegraphics[width=0.6\columnwidth]{./cms12.png}
  \caption{Transverse view of the CMS detector showing the silicon tracker, electromagnetic calorimeter, hadron calorimeter, superconducting solenoid and muon chambers \cite{Chatrchyan:2008aa}.}
  \label{fig:CMS}
\end{figure}


\subsection{The Pixel detector}



\section{The Pixel Cluster Counting (PCC) method}
\label{sec:pcc}

\subsection{Luminosity determination based on PCC}
Pixel cluster counting (PCC)  method is an offline technique for the calculation of instantaneous luminosity $(L_{inst})$ for any LHC run period by counting the number of pixel clusters in the pixel detector (innermost segment of the CMS tracker) in a zero bias event which requires that only two bunches cross at the CMS interaction point. The mean number of pixel clusters $<N_{cl}>$ can be expressed in terms of mean number of pixel clusters per proton-proton interaction  $<N_{cl/pp}>$ and pileup ($\mu$) assuming that clusters do not overlap which is a reasonable assumption given the fine granularity of pixel detector. \\


  
$<N_{cl}> = <N_{cl/pp}> \mu$ \\

$\mu = \frac{<N_{cl}>}{<N_{cl/pp}>}$ \\



The instantaneous luminosity per bunch crossing $L_{b}$ is proportional to the number of proton-proton (pp) interactions per bunch crossing (pileup) and the proportionality constant is the ratio of  LHC orbit frequency f and the pp interaction cross section $\sigma_{pp}$ \cite{CMS-PAS-LUM-12-001}. \\

$L_{b}$ = $\frac{f}{\sigma_{pp}}$ $\mu$  or $L_{b}$ = $\frac{<N_{cl}> \: f }{<N_{cl/pp}> \: \sigma_{pp}}$ \\

The PCC visible cross section $\sigma_{vis}$ which is a calibration constant between pixel clusters rate and luminosity can be defined as \\

$\sigma_{vis}$ = $<N_{cl/pp}>$  $\sigma_{pp}$ \\

Thus, instantaneous luminosity per bunch crossing $L_b$ is given by \\

$L_{b}$ = $\frac{<N_{cl}> \: \sigma_{pp} \: f }{ \:\sigma_{vis} \: \sigma_{pp}}$ or $L_{b}$ = $\frac{<N_{cluster}> \:\: f}{\sigma_{vis}}$ \\

The PCC visible cross section $\sigma_{vis}$ is determined using van der Meer scan method which is described in section 5. Cluster counting can be done over different time periods like per bunch or orbit integrated, 1 Lumi section (23.36s), 4 Lumi nibble (1.46s) and similarly instantaneous luminosity can be calculated over these time periods using PCC method \cite{CMS:2018elu}. \\

\subsection{Run 2018 CMS Data} \\

CMS 2018 data is a collection of runs for each fill in different periods defined during the LHC operation period.

LHC 2018 run is divided into A, B, C and D as described below. \\
\begin{flushleft}
1. Run2018A \\
Run number ranges from 315255 to 316995 \\
Number of runs per period is 146 \\

2. Run2018B\\
Run number ranges from 317080 to 319311\\
Number of runs per period is 132\\

3. Run2018C \\
Run number ranges from 319337 to 320065\\
Number of runs per period is 88 \\

4.Run2018D \\
Run number ranges from 320500 to 323699 \\
Number of runs per period is 262 \\

5. late Run2018D \\
Run number ranges from 320500 to 325175 \\
Number of runs per period is 339 \\
\end{flushleft}

\subsection{Pixel detector Module Selection}
Selection of modules in pixel detector is required because module may experience dynamic inefficiency, bad module chips can stop sending data, chips may not work properly and need reconfiguring. Modules which are observed to function efficiently are considered for luminosity estimation. The module veto lists correspond to late Run2018D that select each detector part barrel layers L2, L3, L4 and forward disks D1, D2, D3 as shown in Fig. 12 and 13. Barrel layer 1 (L1) is not used for luminosity determination \cite{vetolist}. \\

\begin{flushleft}
1. Total number of vetoed modules 1387. \\
2. Number of vetoed modules for L2 1850.\\
3. Number of vetoed modules for L3 1766.\\
4. Number of vetoed modules for L4 1641.\\
5. Number of vetoed modules for D1 1813.\\
6. Number of vetoed modules for D2  1799.\\
7. Number of vetoed modules for D3 1800.\\
\end{flushleft}


\begin{figure}[H]
  \centering
  \includegraphics[width=0.8 \columnwidth]{./PDDisk_R12.png}
  \caption{ \onehalfspacing Diagram showing coordinates of modules in one forward disk of pixel detector having two rings. FPIX 0 correspond to ring 1 and FPIX 1 to ring 2. Yellow regions shows coordinates of vetoed modules while white regions shows coordinates of good modules used for luminosity determination.}
  \label{fig:CMS}
\end{figure}

\begin{figure}[H]
  \centering
  \includegraphics[width=1 \columnwidth]{./bpixL2-L4.png}
  \caption{ \onehalfspacing Diagram showing z coordinate (signed module) and phi coordinate (signed ladder) of modules in three barrel layers (L2, L3 and L4) of pixel detector. Yellow regions shows coordinates of vetoed modules while white regions shows coordinates of good modules used in luminosity determination. }
  \label{fig:CMS}
\end{figure}








\newpage \section{van der Meer calibration}
\label{sec:vdm}
Van der Meer (VdM) scans are done to calculate the visible cross section $\sigma_{vis}$ which is a calibration constant in the relation between rate of a quantity (clusters, coincidences) read out by the detector and the instantaneous luminosity. \\

The instantaneous luminosity when the crossing angle between the colliding bunches is negligible and bunches collide head on is given by the following integral, \\

L = N_1 N_2 f \int^{+\infty}_{-\infty} \rho_1(x,y) \rho_2(x+\Delta x, y+ \Delta y) dx dy \\

where $N_1, N_2$ are the number of protons in the colliding bunches, f is the LHC orbit frequency whose value is 11246Hz,  $\rho_1(x,y)$ and $\rho_2(x+\Delta x,y+\Delta y)$ are the two dimensional particle density distribution functions for each colliding bunch separated by $\Delta x$ and $\Delta y$ in the x and y directions. \\

Assuming complete factorization of particle density distribution function along x and y directions for both the bunches, we get \\

L = N_1 N_2 f \int^{+\infty}_{-\infty} \rho_{1x}(x) \rho_{2x}(x+\Delta x)  dx   \int^{+\infty}_{-\infty} \rho_{1y}(y) \rho_{2y}(y+\Delta y)  dy \\

where $\rho_1(x,y) = \rho_{1x}(x,y) \rho_{1y} (x,y)$ and $\rho_2(x+\Delta x,y + \Delta y) = \rho_{2x}(x+\Delta x,y+\Delta y) \rho_{2y} (x+\Delta,y+\Delta y)$ \\

The VdM scan determines the bunch  overlap integrals $\int \rho_{x1} (x) \rho_{x2} (x+\Delta x) dx$ and  $\int \rho_{y1}(y) \rho_{y2} (y+\Delta y) dy$ by measuring pixel clusters rate as a function of bunch separation $\Delta x$ and $\Delta y$ along x and y directions as shown in Fig. 7. \\

\int \rho_{x1} (x) \rho_{x2} (x+\Delta x) dx = \frac{R_x(0)}{\int R_x(\Delta x)dx} \\

\int \rho_{y1} (y) \rho_{y2} (y + \Delta y) dy = \frac{R_y(0)}{\int R_y(\Delta y)dy} \\

Bunch overlap widths $\Sigma_x$ and $\Sigma_y$ along x and y directions are defined as \\

\Sigma_x = \frac{1}{\sqrt{2 \pi}} \frac{\int R_x(\Delta x)dx}{R_x(0)} \\

\Sigma_y =  \frac{1}{\sqrt{2 \pi}} \frac{\int R_y(\Delta y)dy}{R_y(0)} \\

$\Sigma_x \Sigma_y = \frac{1}{2 \pi} \frac{\int R_x(\Delta x)dx \int R_y(\Delta y) dy}{R_x(0) R_y(0)}$  is the bunch overlapping area as shown in Fig. 8 by the red region.\\

Thus, the expression for instantaneous luminosity is given by \\

L = $\frac{N_1 N_2 f}{2\pi \Sigma_x \Sigma_y}$ \\

and the PCC visible cross section defined in terms of beam parameters and rate is given by \\

\sigma_{vis} = \frac{2 \pi \Sigma_x \Sigma_y R}{N_1 N_2} \\


\begin{figure}[H]
  \centering
  \includegraphics[width=0.7\columnwidth]{./vdmfit.png}
  \caption{ \onehalfspacing Example vdM scans for PCC for BCID 41, from the last scan pair in fill 4954, showing the rate normalized by the product of beam currents as a function of the beam separation $\Delta x$ and $\Delta y$ in the x (left) and y (right) direction, and the fitted curves. The purple curve shows the overall double-Gaussian fit, while the blue, yellow, and green curves show the first and second Gaussian components and the constant component, respectively. The lower panels display the difference between the measured and fitted values
divided by the statistical uncertainty\cite{}.}
  \label{fig:CMS}
\end{figure}


\begin{figure}[H]
  \centering
  \includegraphics[width=\columnwidth]{./vdm_image.png}
  \caption{\onehalfspacing Figure showing two LHC bunches in green and blue approaching each other and colliding giving rise to an overlapping region showing in red whose area is determined during VdM scans.}
  \label{fig:CMS}
\end{figure}








\section{Backgrounds}
\label{sec:bkg}
\subsection{vdM background}

The cluster rate equation is modified in the presence of background events.\\

N = \sigma \:L_{int} + N_{bg} \\

N: total number of signal and background events. \\

$N_{bg}$: Number of background events \\

$N_{bg}$ need to be subtracted during vdM scans for precise determination of $\sigma_{vis}$. These contributions to the measured cluster rate are either subtracted before the fit as measured from noncolliding, unpaired bunches or from periods during which the beams were kept at a large distance, or determined in the fit as a constant term added to the fit function as shown in Fig. 15.

\begin{figure}[H]
  \centering
  \includegraphics[width=0.7\columnwidth]{./PCC_bg.png}
  \caption{ \onehalfspacing Visible cross section $\sigma_{vis}$ with only background subtraction.}
  \label{fig:CMS}
\end{figure}


\subsection{High pileup physics run background}

During high pileup physics runs, background contribution to instantaneous luminosity based on the pixel cluster counting (PCC) method arises from a small out-of-time response (OOTR, or afterglow) that has two types.\\

Type I: fake hits in the bunch crossing after the colliding bunches (signal charge spillover).\\

Type II: Activated material in the detector due to the large radiation doses, exponentially decays for many bunch crossings.\\

Fig. 16 shows the afterglow noise induced by one colliding bunch. Since the distribution is normalized to 1 at the first bin it means the first bin is the colliding bunch. The values at bins greater than 1 are the cluster counts observed for bunches after the colliding bunch divided by the cluster counts observed by the colliding bunch. There are three parts to this distribution: \\

(i) bin 1 is the colliding bunch \\

(ii) bin 2 (empty bunch) is the bunch after colliding bunch, noise observed is due to electronics signal on same pixels from colliding bunch. This is Type I afterglow effect. \\

(iii) bins greater than 2 are all empty bunches. Noise observed is due to "albedo" which is material activated by the radiation of the collisions, it decays exponentially with time just like radioactive material. It produces secondary particles which create hits late in time. This is called Type II afterglow effect. \\

\begin{figure}[H]
  \centering
  \includegraphics[width=0.6\columnwidth]{./SingleBunchAfterglow.png}
  \caption{Afterglow effect produced by a single colliding bunch.}
  \label{fig:LHC}
\end{figure}

 Afterglow effects for various bunches in a beam is shown in Fig. 17 where the intial rising curve correspond to Type I and decay curve correspond to Type II afterglow effect. It was studied in data taken using random triggers (triggers that fire randomly in bunch crossings spread over the whole LHC orbit except for the abort gap), considering the pixel cluster counts in bunch crossings after bunch trains. The resulting correction depends on the LHC filling scheme and for a typical 2011 fill with 1380 bunches corresponds to a subtraction of close to 2.8$\%$ of the integrated luminosity per lumi section. The afterglow corrections for PCC during 2017 run are in the range 2-5 $\%$ for the Type I corrections and 2-3 $\%$ averaged over all active bunches for the Type II corrections \cite{Sirunyan:2759951}.



\begin{figure}[H]
  \centering
  \includegraphics[width=0.5\columnwidth]{./afterglow.png}
  \caption{Results of the pixel afterglow fit. Type I afterglow effect results from fake electronic pulse response after actual bunch crossing and Type II results from activation of detector material shown as exponentially decaying. }
  \label{fig:LHC}
\end{figure}






\section{Systematics}
\label{sec:syst}
(main systematics: vdM, non-linearity, stability, check others in the PAS)


Systematic uncertainties Length scale, Orbit drift, x-y correlations, Beam-beam deflection, dynamic $\beta^*$, Beam current calibration, Ghosts and satellites, Scan to scan variation, Bunch to bunch variation, Cross-detector consistency, Afterglow effects, Linearity, CMS deadtime.  \\

1. Length Scale Calibration:  Value of the separation derived from currents in the corrector magnets, This is calibrated against central tracking system, Compare luminous region information (vertices distribution) versus nominal separation. \\

2. Emittance growth: Beams size are known to increase in size during the fill. Effect is sizable during the ~30 minutes, Bias almost negligible if emittance grows linearly with time and if measurement are made in between X and Y scans. \\

3. Orbit Drift: The beams can slowly drift in the transverse plane; effect is typically small but can be harmful in some cases. Drift of the orbit estimated by BPM measurements taken between scans and extrapolated to IP \\

4. XY-Correlations: The VdM scan method assumes that the bunch proton density function is factorizable into independent x- and y-dependent terms. However, this assumption is not strictly valid, and can lead to a biased estimate of the beam overlap area. In order to measure this effect, special beam imaging scans are conducted. In these scans, the measured vertex position distributions are used to derive the bunch proton densities, which can then be used to estimate a correction to the visible cross section \\

5. Beam-Beam Effects: Dipolar kick (beam-beam deflection):

 Repulsive force deflects beams affecting nominal separation depends on the separation itself, the beam width and the current
 
 Quadrupolar (de)focusing (dynamic b):  Effective beam width modified depending on the separation Peak luminosity (rate) affected \\



\section{Results}
\label{sec:results}

\subsection{PCC count and luminosity}

\begin{figure}[H]
  \centering
  \includegraphics[width=1\columnwidth]{./crop.png}
  \caption{PCC luminosity ($\mu b^{-1}$) and PCC/hfoc luminosity ratio for Run2018A using late Run2018D veto list for pixel detector modules.}
  \label{fig:CMS}
\end{figure}


\begin{figure}[H]
  \centering
  \includegraphics[width=1\columnwidth]{./317438(1).png}
  \caption{PCC luminosity ($\mu b^{-1}$) and PCC/hfoc luminosity ratio for Run2018B using late Run2018D veto list for pixel detector modules.}
  \label{fig:CMS}
\end{figure}


\begin{figure}[H]
  \centering
  \includegraphics[width=1\columnwidth]{./319659.png}
  \caption{PCC luminosity ($\mu b^{-1}$) and PCC/hfoc luminosity ratio for Run2018C using late Run2018D veto list for pixel detector modules.}
  \label{fig:CMS}
\end{figure}


PCC cluster counts and calculated integrated luminosity after applying luminosity calibration (visible cross section) for Run2018 A, B and C are shown below.

 \begin{figure}[H]
  \centering
  \includegraphics[width=0.52\columnwidth]{./ls_lumi.png}
  \caption{PCC count as a function of luminosity section (1 lumi section is 23.36s) for Run2018A.}
  \label{fig:CMS}
\end{figure}


\begin{figure}[H]
  \centering
  \includegraphics[width=0.5\columnwidth]{./ls_lumi_2018B.png}
  \caption{PCC count as a function of lumi section for Run2018B.}
  \label{fig:CMS}
\end{figure}

\begin{figure}[H]
  \centering
  \includegraphics[width=0.5\columnwidth]{./ls_lumi_2018C.png}
  \caption{PCC count as a function of lumi section for Run2018C.}
  \label{fig:CMS}
\end{figure}


\begin{figure}[H]
  \centering
  \includegraphics[width=0.52\columnwidth]{./runs.png}
  \caption{PCC integrated luminosity ($\mu b^{-1}$) as a function of run number for Run2018A.}
  \label{fig:CMS}
\end{figure}


\begin{figure}[H]
  \centering
  \includegraphics[width=0.52\columnwidth]{./runs_2018B.png}
  \caption{PCC integrated luminosity ($\mu b^{-1}$) as a function of run number for Run2018B.}
  \label{fig:CMS}
\end{figure}


\begin{figure}[H]
  \centering
  \includegraphics[width=0.52\columnwidth]{./runs_2018C.png}
  \caption{PCC integrated luminosity ($\mu b^{-1}$) as a function of run number for Run2018C.}
  \label{fig:CMS}
\end{figure}


\subsection{Stability of PCC luminosity for Run 2018 data}

The stability of run 2018 CMS integrated luminosity for Phase I pixel detector calculated using PCC method is investigated by employing late Run2018D veto list and including afterglow effects. New module selections will be determined and used for 2018 luminosity determination. 
\begin{figure}[H]
  \centering
  \includegraphics[width=1\columnwidth]{./t25ld-kgsjf.png}
  \caption{Left: Luminosity ratios for various subdetectors L2, L3, L4, D1, D2, D3 of pixel detector as a function of lumi section for Run2018A. Right: X Profile of luminosity ratios vs lumi section graph for various subdetectors L2, L3, L4, D1, D2, D3 of pixel detector showing luminosity fraction as a function of lumi section (1 lumi section is 23.36s) for Run2018A.}
  \label{fig:CMS}
\end{figure}


\begin{figure}[H]
  \centering
  \includegraphics[width=1\columnwidth]{./tjag3-rger6.png}
  \caption{Left: Luminosity ratios for various subdetectors L2, L3, L4, D1, D2, D3 of pixel detector as a function of lumi section for Run2018B. Right: X Profile of luminosity ratios vs lumi section graph for various subdetectors L2, L3, L4, D1, D2, D3 of pixel detector showing luminosity fraction as a function of lumi section for Run2018B. }
  \label{fig:CMS}
\end{figure}



\begin{figure}[H]
  \centering
  \includegraphics[width=1\columnwidth]{./tfkx3-uyzgt.png}
  \caption{Left: Luminosity ratios for various subdetectors L2, L3, L4, D1, D2, D3 of pixel detector as a function of lumi section for Run2018C. Right: X Profile of luminosity ratios vs lumi section graph for various subdetectors L2, L3, L4, D1, D2, D3 of pixel detector showing luminosity fraction as a function of lumi section for Run2018C \cite{lumidpg}.}
  \label{fig:CMS}
\end{figure}






 \section{The TEPX upgrade for Phase II HL-LHC}
\label{sec:tepx}

The High Luminosity (HL)-LHC will increase instantaneous luminosity to unprecedented value of $7.5 \times 10^{34} cm^{-2} s^{-1}$ which corresponds to 200 proton-proton collisions per bunch crossing (pileup). Run 2 pixel detector will not be able to handle the extreme radiation environment, resolve nearby particle tracks and operate properly to give a reliable estimate of the instantaneous luminosity for high pileup values. That is why it will be replaced by a new pixel detector which will be composed of three subdetectors: tracker barrel pixel detector (TBPX), tracker forward pixel detector (TFPX) and tracker endcap pixel detector (TEPX). TEPX will have better radiation tolerance, increased granularity, improved two-track separation, improved estimation of hit rate and statistical precision, extended tracking acceptance $|\eta|=4$ with Disk 4 Ring 1 operating as an independent luminometer \cite{Klein:2017nke}. \\


\subsection{TEPX detector for Phase II}

Tracker endcap pixel detector (TEPX) consists of four double disks per side (-Z and +Z) with each double disk containing five rings as shown in Fig. 25 having 20, 28, 36, 44 and 48 modules respectively. One double disk has four surfaces with +Z side containing modules with even module number in front layers (L1 $\&$ L2) and modules with odd module number in back layers (L3 $\&$ L4) from Ring 1 to Ring 4 and for Ring 5, modules with odd module number in front layers and modules with even module number in back layers as shown in Fig. 26. For -Z side, four surfaces contain modules with odd module number in front layer and modules with even module number in back layer from Ring 1 to Ring 4 and for Ring 5, modules with even module number in front layer and modules with odd module number in back layer. \\


\begin{figure}[H]
  \centering
  \includegraphics[width=1 \columnwidth]{./tepx_tt.png}
  \caption{ \onehalfspacing Left: A layout of the CMS Phase II inner tracker showing four TEPX disks, eight TFPX disks and four barrel layers. Right: Diagram showing one double disk of TEPX with five rings \cite{Klein:2017nke}.}
  \label{fig:CMS}
\end{figure}


\begin{figure}[H]
  \centering
  \includegraphics[width=1 \columnwidth]{./fourlayers.png}
  \caption{ \onehalfspacing Fours layers of one double disk of TEPX showing module arrangement in rings for each layer. Ring 1, 2, 3, 4 and 5 consists of 20, 28, 36, 44 and 48 modules respectively. }
  \label{fig:CMS}
\end{figure}

\subsection{Phase II CMS simulation samples}

\onelinespacing Simulated data samples for Phase II include full CMS detector description and uses official CMS software (CMSSW) version $10-6-0-patch2$ which calls GEANT4 for particle and energy deposit simulation as well as for reconstruction \cite{Agostinelli:2002hh}. These samples contain single-neutrino event overlaid with a variable number of minimum-bias events (events with any amount of real energy detected in CMS) to simulate different pileup values. The statistics for samples with average pileup values from 0.5 to 2 is 500000 events per step and for average pileup values between 10 and 200, statistics is 100000. \\

\subsection{Luminosity determination using TEPX - counting clusters/coincidences}

Instantaneous luminosity determination using PCC method for Phase II HL-LHC will be based on counting the number of clusters in TEPX Disk 4 Ring 1. The innermost ring of the last disk of TEPX (D4R1) is located at 2.65 m away from the interaction point that is beyond the tracking acceptance ($|\eta| = 4$) and as this region has few tracking points, it can be solely used for the purpose of luminosity measurement by using the full available trigger rate and bandwidth.\\

A new algorithm for luminosity determination based on counting two fold coincidences is proposed that applies selections using dr and $d\phi$ variables to minimize non-linearity for high pileup values as these variables take into account track angle and its curvature. Two fold coincidences are those hits which are created by the module overlap regions between various layers of one TEPX double disk. Two fold coincidences are better way to distinguish between a real hit and random electrical noise. They are more likely to be real hit than random electrical noise. Two fold coincidence in $\phi$ will involve modules overlapping in the same ring in front and back layers of one double disk as shown in top part of Fig. 27 and Fig. 28 while two fold coincidence in r will require modules overlapping between successive rings in the front (L1 $\&$ L2) and back layers (L3 $\&$ L4) of one double disk as shown in bottom part of Fig. 27. Luminosity calculated based on counting coincidences has an advantage over clusters that afterglow effects are tiny in the case of coincidences  \cite{Collaboration:2706512} \cite{brilsim} \cite{brilsim1}\cite{brilsim2}.\\



\begin{figure}[H]
  \centering
  \includegraphics[width=0.7\columnwidth]{./2foldinrphi.png}
  \caption{ \onehalfspacing Diagram showing modules overlap between the front (L1 \& L2) and back (L3 \& L4) layers of one double disk of TEPX that creates two fold coincidences in $\phi$ and r.}
  \label{fig:CMS}
\end{figure}

\begin{figure}[H]
  \centering
  \includegraphics[width=0.6\columnwidth]{./23coin.png}
  \caption{\onehalfspacing Example of two and threefold coincidence regions on a portion of a single TEPX disk.}
  \label{fig:CMS}
\end{figure}

\subsection{TEPX linearity}

Linearity is one of the systematic uncertainty in the calculation of instantaneous luminosity and PCC visible cross section $\sigma_{vis}$. A linear relation between the number of clusters and pileup (PU) imply that \\

$<N_{cl/pp}> = \frac{<N_{cl}>}{\mu} = \frac{\sigma_{vis}}{\sigma_{pp}}$ \\

$<N_{cl}> =  \frac{\sigma_{vis}}{\sigma_{pp}} \mu $ \\

Linearity indicates that the PCC visible cross section $\sigma_{vis}$ does not depend on the per bunch instantaneous luminosity (ideal luminometer). In ideal scenario, $\sigma_{vis}$ is not dependent on pileup, but this a potential problem with the luminometer. Linearity results for simulated TEPX clusters and coincidences from low to high pileup values are shown from Fig. 29 to Fig. 40. TEPX luminometer proposed for Phase II shows excellent linearity over entire pileup range. A non-linear relation $<N_{cluster}> = \alpha (PU)^{\gamma}, \gamma \neq 1$ would add non-linear terms in the rate equation $R = \sigma_{vis} L_{inst}$  and cause $\sigma_{vis}$ to vary with the per bunch instantaneous luminosity and pileup values.


\begin{figure}[H]
  \centering
  \includegraphics[width=1\columnwidth]{./totalclusters.png}
  \caption{\onehalfspacing Left: Simulated mean number of clusters for all entire TEPX detector as a function of pileup. A line is fitted between pileup values of 0 and 2, and then extrapolated up to a pileup of 200. Right: Deviation from linearity for clusters for entire TEPX detector. The non-linearity is calculated as the relative difference between the data points and the values of the fit function at the respective pileup value. Non-linearity is within 1 \% for entire pileup range. Pileup 200 corresponds to High Luminosity (HL)-LHC environment.}
  \label{fig:CMS}
\end{figure}



\begin{figure}[H]
  \centering
  \includegraphics[width=1\columnwidth]{./clustersD4R1.png}
  \caption{\onehalfspacing Left: Simulated mean number of clusters for TEPX Disk 4 Ring 1 as a function of pileup. Right: Deviation from linearity for clusters for TEPX Disk 4 Ring 1. The non-linearity is calculated as the relative difference between the data points and the values of the fit function at the respective pileup value.}
  \label{fig:CMS}
\end{figure}



\begin{figure}[H]
  \centering
  \includegraphics[width=1 \columnwidth]{./clustersperdisk+z.png}
  \caption{Left: Simulated mean number of clusters for +z side TEPX disks as a function of pileup. Right: Deviation from linearity for clusters for +z side TEPX disks.}
  \label{fig:CMS}
\end{figure}




\begin{figure}[H]
  \centering
  \includegraphics[width=1\columnwidth]{./clustersperringD+4.png}
  \caption{Left: Simulated mean number of clusters for +z side TEPX Disk 4 all rings as a function of pileup. Ring 1 has highest slope and Ring 5 has least slope. Right: Deviation from linearity for clusters for TEPX Disk +4 all rings. Non-linearity is within $1\%$ for all rings over entire pileup range.}
  \label{fig:CMS}
\end{figure}


\begin{figure}[H]
  \centering
  \includegraphics[width=1 \columnwidth]{./totalcoincidences.png}
  \caption{Left: Simulated mean number of coincidences in $\phi$ and r for all entire TEPX detector as a function of pileup. A line is fitted between pileup values of 0 and 2, and then extrapolated up to a pileup of 200. Right: Deviation from linearity for coincidence in $\phi$ and r for entire TEPX detector. The non-linearity is calculated as the relative difference between the data points and the values of the fit function at the respective pileup value. Non-linearity is within 1 \% for entire pileup range. Pileup 200 corresponds to High Luminosity (HL)-LHC environment.}
  \label{fig:CMS}
\end{figure}


\begin{figure}[H]
  \centering
  \includegraphics[width=1\columnwidth]{./totalcoincidencesD4R1.png}
  \caption{Left: Simulated mean number of coincidences in $\phi$ and r for TEPX Disk 4 Ring 1 as a function of pileup. Right: Deviation from linearity for coincidences in $\phi$ and r for TEPX Disk 4 Ring 1. The non-linearity is calculated as the relative difference between the data points and the values of the fit function at the respective pileup value.}
  \label{fig:CMS}
\end{figure}





\begin{figure}[H]
  \centering
  \includegraphics[width=1\columnwidth]{./coincidencesperdisk+z.png}
  \caption{Left: Simulated mean number of coincidences in $\phi$ and r for +z side TEPX disks as a function of pileup. Right: Deviation from linearity for coincidences in $\phi$ and r for +z side TEPX disks.}
  \label{fig:CMS}
\end{figure}


\begin{figure}[H]
  \centering
  \includegraphics[width=1\columnwidth]{./coincidencesinphiD4R1z+.png}
  \caption{Left: Simulated mean number of coincidences in $\phi$ for TEPX +z side Disk 4 Ring 1 as a function of pileup. Right: Deviation from linearity for coincidences in $\phi$ for TEPX +z side Disk 4 Ring 1. The non-linearity is calculated as the relative difference between the data points and the values of the fit function at the respective pileup value.}
  \label{fig:CMS}
\end{figure}



\begin{figure}[H]
  \centering
  \includegraphics[width=1\columnwidth]{./coincidencesinrD4R1z+.png}
  \caption{Left: Simulated mean number of coincidences in r for TEPX +z side Disk 4 Ring 1 as a function of pileup. Right: Deviation from linearity for coincidences in r for TEPX +z side Disk 4 Ring 1. The non-linearity is calculated as the relative difference between the data points and the values of the fit function at the respective pileup value.}
  \label{fig:CMS}
\end{figure}





\begin{figure}[H]
  \centering
  \includegraphics[width=1\columnwidth]{./coincidencesperringD+4.png}
  \caption{Left: Simulated mean number of coincidences in $\phi$ and r for +z side TEPX Disk 4 per ring as a function of pileup. Ring 1 has highest slope and Ring 5 has least slope. Right: Deviation from linearity for coincidences in $\phi$ and r for +z side TEPX Disk 4 per ring. Non-linearity is within 1\% for all rings over entire pileup range.}
  \label{fig:CMS}
\end{figure}








\begin{figure}[H]
  \centering
  \includegraphics[width=1\columnwidth]{./coincidencesinphiperringD+4.png}
  \caption{Left: Simulated mean number of coincidences in $\phi$ for +z side TEPX Disk 4 per ring as a function of pileup. Ring 1 has highest slope and Ring 5 has least slope. Right: Deviation from linearity for coincidences in $\phi$ for +z side TEPX Disk 4 per ring. Non-linearity is within 1\% for all rings over entire pileup range.}
  \label{fig:CMS}
\end{figure}





\begin{figure}[H]
  \centering
  \includegraphics[width=1\columnwidth]{./coincidencesinrperringD+4.png}
  \caption{Left: Simulated mean number of coincidences in r for +z side TEPX Disk 4 per ring as a function of pileup. Ring 1 has highest slope and Ring 5 has least slope. Right: Deviation from linearity for coincidences in r for +z side TEPX Disk 4 per ring. Non-linearity is within $1\%$ for all rings over entire pileup range.}
  \label{fig:CMS}
\end{figure}


\subsection{Statistical precision for TEPX luminometer}
Statistical precision is another uncertainty that need to be considered for precise luminosity measurement. It must be kept minimal to achieve 1 $\%$ accuracy for HL-LHC luminosity measurement. \\

Relative statistical uncertainty in $\%$ = $\frac{\sqrt{N}}{N} \times 100$ \\

where N = (Number of counts per event)$\times$(Trigger Frequency)$\times$(Time Integration period). \\

\newpage

\clearpage\newpage

\begin{flushleft} 
Table 1: Expected statistical precision (in $\%$) in head-on collisions during typical vdM conditions with pileup of 0.5 for TEPX clusters and two fold coincidences over different integration time period.
\end{flushleft} 
\begin{center}
\scalebox{1}{
\begin{tabular}{|l | c | c | c |c|}
\hline
 vdM (PU 0.5) & Readout Rate (kHz) &1 bx, 1s & 1 bx, 30s & 100 bx, 30s\\
\hline
TEPXD4R1 Clusters&1000&1.82&0.332 & 0.0332\\
\hline
TEPXD4R1 2x Coincidences in phi &1000&5.98&1.09 & 0.109\\
\hline
TEPXD4R1 2x Coincidences in r &1000&11.1&2.02 & 0.202\\
\hline
TEPXD4R1 2x Coincidences &1000&5.27&0.962 & 0.0962\\
\hline
TEPX Clusters&500&0.709&0.129 & 0.0129\\
\hline
TEPX 2x Coincidences in phi &500&2.65&0.485 & 0.0485\\
\hline
TEPX 2x Coincidences in r &500&4.59&0.838 & 0.0838\\
\hline
TEPX 2x Coincidences &500&2.3&0.42 & 0.042\\
\hline
\end{tabular}}
\end{center}

\begin{flushleft} 
  Table 2: Expected statistical precision (in $\%$) in head-on collisions during physics conditions with pileup of 200 for TEPX clusters and two fold coincidences over different integration time period.
  \end{flushleft} 
\begin{center}
\scalebox{1}{
\begin{tabular}{|l | c | c | c |c |}
\hline
Algorithm & Readout Rate (kHz)& 1 bx, 1s &2500 bx, 1s   & 2500 bx, 1 LS \\
\hline
TEPXD4R1 Clusters&825&0.1&0.002&0.000414\\
\hline
TEPXD4R1 2x Coincidences in phi &825&0.329&0.00659&0.00136\\
\hline
TEPXD4R1 2x Coincidences in r &825&0.61&0.0122&0.00252\\
\hline
TEPXD4R1 2x Coincidences  &825&0.29&0.0058&0.0012\\
\hline
TEPX Clusters &75&0.0915&0.00183&0.000379\\
\hline
TEPX 2x Coincidences in phi &75&0.343&0.00685&0.00142\\
\hline
TEPX 2x Coincidences in r &75&0.593&0.0119&0.00245\\
\hline
TEPX 2x Coincidences &75&0.297&0.00594&0.00123\\
\hline
\end{tabular}}
\end{center}

\newpage



\section{Conclusions and outlook} 
\label{sec:conclusion}
sumarize the method and results, and TEPX upgrade work and plans



\cleardoublepage
\onehalfspacing
\bibliographystyle{unsrt}
\bibliography{paper}

\end{document}




