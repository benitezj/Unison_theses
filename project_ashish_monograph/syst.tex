\section{Systematics}
\label{sec:syst}
(main systematics: vdM, non-linearity, stability, check others in the PAS) \\

Systematic uncertainties in the beam overlap widths along x and y directions measured during vdM scans, $\sigma_{vis}$ and CMS luminosity can arise from length scale calibration, orbit drift, XY correlations of beam proton density distribution function, emittance growth, beam-beam deflection, dynamic $\beta$ which tells us whether bunches are narrow or wide, beam current calibration, ghosts and satellites, scan to scan variation and bunch to bunch variation. Systematic uncertainty during physics runs (high pileup) can be due to cross-detector consistency, afterglow effects, linearity and CMS deadtime. \\

\textbf{Systematics uncertainties during vdM scans or in $\bf{\sigma_{vis}}$:} \\

1. Length scale calibration: Acurracy of the beam positions from the LHC magnetic currents is limited that is why length scale calibration is done in which beams are moved in forward and backward directions at constant separation. A linear fit between the measured position and nominal position is plotted and deviation of the slope of this curve from one is used as correction. \\

2. Emittance growth: Beam size is known to increase during the fill which affects the cluster rate profile amplitude, its width and luminosity. Their variation is studied as a function of time and a linear fit is performed to obtain correction. Effect is sizable during a period of 30 minutes. \\

3. Orbit drift: Beams can slowly drift from their nominal positions in the transverse plane which causes beam separation to vary over a period of vdM scan step. LHC beam position monitoring systems are used to correct for beam separation. \\

4. XY correlations: The vdM scan method assumes that the shape of the bunch proton density function is factorizable into independent X and Y dependent terms which is not fully true and can lead to a biased estimate of the bunch overlap area $\Sigma_x \Sigma_y$. In order to measure this effect, special beam imaging scans are conducted in which one beam is fixed and other beam is moved across. In these scans, the measured vertex position distributions are used to derive the beam proton densities where Super Double Gaussian function is used as best fit and correction to the visible cross section $\sigma_{vis}$ is estimated. \\

5. Beam-Beam Effects: \\

Dipolar kick (beam-beam deflection): Repulsive force due to beam's electric field deflects other beam which affect nominal separation, the beam width and beam current. \\
Quadrupolar defocusing (dynamic $\beta$): Beam can get defocused by other beam's magnetic quadrupole field. Effective beam width is modified depending on the separation which in turn affects the peak instantaneous luminosity value because collision rate is decreased by 10 \%. \\

\textbf{Systematics uncertaintities during physics runs (high pileup)}:\\

6. Non-Linearity: extrapolation of $\sigma_{vis}$ from vdM scans to high pileup physics data-taking periods can cause $\sigma_{vis}$ to vary with per bunch instantaneous luminosity and pileup if the relation between luminosity and pileup is not linear.                               \\

7. Stability: Instability can arise from variation of measured cluster rate over time for different subdetectors in pixel detector and change of detector conditions over time for example detector noise, aging effects, radiation damage, etc. Results for PCC luminosity stability for 2018 data is described in the next section.                               \\

8. Afterglow effects: A small out of time response appears in unfilled bunches after colliding bunches. These effects are described in detail in the previous section. \\


9. Cross-detector comparisons: After corrections of out-of-time effects (afterglow), stability and linearity are applied, luminosity measurement from various detector like BCM1F, PLT and HFOC can be compared to do very precise luminosity determination by taking into account exact corrections \cite{CMS-PAS-LUM-18-002}. 



