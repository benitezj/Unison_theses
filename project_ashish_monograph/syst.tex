\section{Systematics}
\label{sec:syst}
(main systematics: vdM, non-linearity, stability, check others in the PAS)


Systematic uncertainties Length scale, Orbit drift, x-y correlations, Beam-beam deflection, dynamic $\beta^*$, Beam current calibration, Ghosts and satellites, Scan to scan variation, Bunch to bunch variation, Cross-detector consistency, Afterglow effects, Linearity, CMS deadtime.  \\

1. Length Scale Calibration:  Value of the separation derived from currents in the corrector magnets, This is calibrated against central tracking system, Compare luminous region information (vertices distribution) versus nominal separation. \\

2. Emittance growth: Beams size are known to increase in size during the fill. Effect is sizable during the ~30 minutes, Bias almost negligible if emittance grows linearly with time and if measurement are made in between X and Y scans. \\

3. Orbit Drift: The beams can slowly drift in the transverse plane; effect is typically small but can be harmful in some cases. Drift of the orbit estimated by BPM measurements taken between scans and extrapolated to IP \\

4. XY-Correlations: The VdM scan method assumes that the bunch proton density function is factorizable into independent x- and y-dependent terms. However, this assumption is not strictly valid, and can lead to a biased estimate of the beam overlap area. In order to measure this effect, special beam imaging scans are conducted. In these scans, the measured vertex position distributions are used to derive the bunch proton densities, which can then be used to estimate a correction to the visible cross section \\

5. Beam-Beam Effects: Dipolar kick (beam-beam deflection):

 Repulsive force deflects beams affecting nominal separation depends on the separation itself, the beam width and the current
 
 Quadrupolar (de)focusing (dynamic b):  Effective beam width modified depending on the separation Peak luminosity (rate) affected \\

