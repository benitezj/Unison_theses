\section{Systematics}
\label{sec:syst}
(main systematics: vdM, non-linearity, stability, check others in the PAS) \\


Systematic uncertainties in the bunch overlap widths along x and y directions measured during vdM scans, $\sigma_{vis}$ and CMS luminosity can arise from length scale calibration, orbit drift, x-y correlations of bunch proton density function, emittance growth, bunch-bunch deflection, dynamic $\beta$ which tells us whether bunches are narrow or wide, bunch current calibration, ghosts and satellites, scan to scan variation and bunch to bunch variation. Systematic uncertainty during physics runs (high pileup) can be due to cross-detector consistency, afterglow effects, linearity and CMS deadtime.  \\

1. Length scale calibration:  Value of the separation derived from currents in the corrector magnets, calibrated against central tracking system and compare luminous region information (vertices distribution) versus nominal separation. \\

2. Emittance growth: Bunches size are known to increase in size during the fill. Effect is sizable during the ~30 minutes, Bias almost negligible if emittance grows linearly with time and if measurement are made in between X and Y scans. \\

3. Orbit drift: The beams can slowly drift in the transverse plane. This effect is typically small but can be harmful in some cases. Drift of the orbit estimated by BPM measurements taken between scans and extrapolated to interaction point. \\

4. xy-correlations: The vdM scan method assumes that the bunch proton density function is factorizable into independent x- and y-dependent terms which is not fully true and can lead to a biased estimate of the bunch overlap area. In order to measure this effect, special beam imaging scans are conducted. In these scans, the measured vertex position distributions are used to derive the bunch proton densities, which can then be used to estimate a correction to the visible cross section $\sigma_{vis}$. \\

5. Linearity: \\

6. Stability:  \\

7. Bunch-Bunch Effects: \\

Dipolar kick (bunch-bunch deflection): Repulsive force deflects beams affecting nominal separation depends on the separation itself, the beam width and the current. \\
 
 Quadrupolar (de)focusing (dynamic $\beta$): Effective beam width is modified depending on the separation which in turn affects the peak luminosity (rate) value. \cite{CMS-PAS-LUM-18-002} \\

