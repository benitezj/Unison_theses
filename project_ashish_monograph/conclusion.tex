\newpage \section{Summary and Plans} 
\label{sec:conclusion}

In the first part, we processed 2018 CMS data set for period A, B and C to compute luminosity ratios for various subdetectors (L2, L3, L4, D1, D2, D3) of pixel detector. Luminosity determined for pixel subdetectors L3 and F1 shows large instabilities for large lumi section values. We will carry out a detailed study to understand these fluctuations/instabilities observed in the luminosity ratios using new module selections. We will also study vdM background, systematics uncertainties in luminosity measurement and determination of $\sigma_{vis}$ for other luminosity measurement device used during Run 2018 like HFOC.\\

In the second part, we performed a linearity study for Phase II luminosity measurement device Tracker Endcap Pixel Detector (TEPX). Simulated samples from low to high pileup values mimicking Phase II HL-LHC conditions were produced using full CMS detector description and used them to study the behaviour of clusters under different pileup conditions. Different selections on separation between variables defined using cluster position dr and $d\phi$ were applied to study the behaviour of two fold coincidences with pileup to take into account track angle and its curvature. We used different cut values for different TEPX disks and rings to obtain a behaviour closest to linearity for two fold coincidences in $\phi$ and implemented a similar procedure for two fold coincidences in r. New tighter selections dependent on TEPX disks and rings based on truth-match cluster distributions removed physics noise at high pileup. Residual non-linearity for TEPX clusters is with 0.5 \% and for two fold coincidences in $\phi$, r and combined is within 1 \% for all TEPX disks, rings and entire TEPX detector except Ring 5 up to pileup 200.\\
 
Statistical precision is also calculated for TEPX clusters and coincidences during typical vdM and high pileup physics conditions over different integration time period. For HL-LHC conditions which correspond to pileup 200, the statistical precision which is an important uncertainty for any luminometer is found to be less than 1\%.
